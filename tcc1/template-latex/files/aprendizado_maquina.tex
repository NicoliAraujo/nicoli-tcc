%!TEX root = ../sbc-template.tex

Aprendizado de máquina, em inglês \emph{Machine Learning}, é o estudo sistemático de algoritmos e sistemas que são capazes de melhorar sua performance com a experiência. Um algoritmo que tem este comportamento é aquele capaz de aprender a partir de dados, assim como humanos e outros animais. Estes, ao se depararem com determinada situação (ou tarefa), costumam tentar lembrar se da última vez em que estiveram em uma situação parecida, tentaram alguma ação que pode ter dado certo -- então deve ser repetida, ou errado -- então deve ser esquecida \cite{marsland2015machine}, \cite{goodfellow2016deep}, \cite{flach2012machine}. Com base neste modelo de aprendizado que considera a adaptação às situações apresentadas, um algoritmo que aprende a partir da experiência $E$ quanto a um conjunto de tarefas $T$ e medida de performance $P$, se sua performance nas tarefas em $T$, medida por $P$, melhora com a experiência $E$ \cite{mitchell1997machine}. Ao inferir um algoritmo de \emph{machine learning} para desenvolver determinada tarefa, busca-se um modelo, ou seja, uma função que mapeie as instâncias do espaço de entrada para o de saída \cite{flach2012machine}. Estes modelos podem ser agrupados em diferentes categorias ao se considerar o tipo de aprendizado e de saída desejada para o algoritmo. Na Figura \ref{fig:ml_algorithms} está uma visão geral dos modelos de algoritmos de \emph{machine learning} e suas subdivisões.

\begin{sidewaysfigure}
	\includegraphics[width=\linewidth]{img/machinelearningalgorithms.png}
	\caption{Mapa mental dos algoritmos de \emph{Machine Learning} organizados por área e sub-área.}
	\label{fig:ml_algorithms}
\end{sidewaysfigure}


Quanto ao tipo de aprendizado, as tarefas de \emph{machine learning} podem ser agrupadas em três tipos diferentes, a depender da presença e do tipo de resposta dada ao algoritmo quanto à acurácia de suas saídas. No aprendizado supervisionado o algoritmo deve aprender a inferir valores a partir de dados rotulados, ou seja, que têm seus valores de saída conhecidos, apresentados na fase de treinamento, a exemplo das máquinas de vetores de suporte, redes neurais artificiais \emph{feedforward}, regressão linear e contínua, etc. Já no aprendizado não-supervisionado, o algoritmo deve inferir padrões e estruturas a partir de dados não tabelados, como é o caso de modelos como \emph{k-means}, redes neurais \emph{deep belief} e detecção de anomalia. Por fim, no aprendizado por reforço o algoritmo não recebe nem dados nem rótulos, e deve aprender a partir das recompensas positivas ou negativas dadas por ações que modifiquem o ambiente de maneira satisfatória ou não \cite{flach2012machine}.

Já quanto ao tipo de saída desejado, os problemas podem ser atacados são a classificação, regressão, transcrição, tradução automática, detecção de anomalia, síntese e amostragem. As principais tarefas que podem ser endereçadas utilizando aprendizado supervisionado são a classificação e a regressão \cite{flach2012machine}. Um algoritmo proposto a uma tarefa de classificação deve especificar cada entrada $x$ como pertencente a uma dentre $k$ categoritas pré-determinadas, produzindo uma saída $y=f(x)$ tal que a função $f$ é definida como $f: \mathds{R}^n \rightarrow \{1, \ldots, k\}$, ou seja, $f$ mapeia sequências de números reais  $x$ de dimensão $n$ para um valor $y$ do meio de $k$ possibilidades \cite{goodfellow2016deep}. Dentre as tarefas de classificação estão o reconhecimento de objetos em uma imagem, determinar se um indivíduo será ou não vítima de determinada doença, se sobreviverá ou não a determinado acidente, etc. Uma tarefa de regressão envolve aprender uma função de valor real a partir de uma entrada \cite{flach2012machine}. Assim, a saída $y=f(x)$ é dada pela função $f: \mathds{R}^n \rightarrow \mathds{R}$, ou seja, $f$ mapeia uma entrada multidimensional $x$ para um valor $y$ real  \cite{goodfellow2016deep}. Algumas tarefas de regressão envolvem a previsão de preços de um mercado de ações, a determinação do risco do seguro para um carro, do volume diário de precipitação em determinada cidade, etc.
