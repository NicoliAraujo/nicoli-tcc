%!TEX root = ../sbc-template.tex
%\todo[inline]{Resultado das métricas após o teste dos modelos.}

Inicialmente, fez-se uso da função de ativação \emph{ReLU} para as camadas de saída dos modelos propostos. Os resultados obtidos estão expostos na Tabela \ref{tab:results_relu} a seguir:
\begin{table}
     \caption{Resultados preliminares do treino e teste dos modelos propostos utilizando \emph{ReLU} na camada de saída.}
     \label{tab:results_relu}
     \centering
     \begin{tabular}{l l l l}
          \toprule
          Modelo & Épocas &RMSE \\
          \midrule
          LeNet & 95 & $41.08$ \\
          AlexNet & 55 & $41.96$
          \bottomrule
     \end{tabular}
\end{table}

Os RMSEs obtidos na fase de testes para as redes que utilizam \emph{Leaky ReLU} como função de ativação na saída e taxa de aprendizado inicial de $10^{-3}$ podem ser observados na Tabela \ref{tab:results_leaky}

\begin{table}[h!]
     \caption{Resultados preliminares do treino e teste dos modelos propostos utilizando \emph{Leaky ReLU} na camada de saída.}
     \label{tab:results_leaky}
     \centering
     \begin{tabular}{l l}
          \toprule
          Modelo & Épocas & RMSE \\
          \midrule
          LeNet & 12 & $41.55$ \\
          AlexNet & 6 & $14.38$
          \bottomrule
     \end{tabular}
\end{table}
