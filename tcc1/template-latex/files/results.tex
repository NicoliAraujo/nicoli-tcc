%!TEX root = ../sbc-template.tex
%\todo[inline]{Resultado das métricas após o teste dos modelos.}

Considerando a abordagem descrita na solução proposta, os resultados preliminares da execução das CNNs aplicadas ao problema de estimação de idade a partir de uma imagem de face são apresentados a seguir.

Conforme mencionado na Seção \ref{subsec:modelos}, os primeiros treinamentos e testes compreenderam as arquiteturas canônicas LeNet e AlexNet com função de ativação \emph{ReLU} na camada de saída. Obedecendo ao método de validação cruzada \emph{holdout} previamente mencionado, os resultados da etapa de teste foram obtidos, detalhados na Tabela \ref{tab:results_relu}.


\begin{table}[h!]
     \caption{Resultados preliminares do treino e teste dos modelos propostos utilizando \emph{ReLU} na camada de saída.}
     \label{tab:results_relu}
     \centering
     \begin{tabular}{l l l}
          \toprule
          Modelo & Épocas &RMSE \\
          \midrule
          LeNet & $95$ & $41.08$ \\
          AlexNet & $55$ & $41.96$\\
          \bottomrule
     \end{tabular}
\end{table}

Ao observar as previsões realizadas para exemplos individuais, percebeu-se uma tendência destas redes após treinamento em preverem valores baixos, indicando possivelmente \emph{underfitting} em virtude do \emph{ReLU dying problem} \cite{djork2015elus, dabal2018elus}. Como alternativa, estes autores sugerem  utilizar variantes da \emph{ReLU} que não exibam saídas nulas, diferentes estratégias de inicialização e regularização de pesos e \emph{batches}, entre outras. Como exposto na Seção \ref{subsec:modelos}, adotou-se a \emph{Leaky ReLU} como função de ativação da camada de saída. Os resultados deste treinamento estão expostos na Tabela \ref{tab:results_leaky}.

\begin{table}[h!]
     \caption{Resultados preliminares do treino e teste dos modelos propostos utilizando \emph{Leaky ReLU} na camada de saída.}
     \label{tab:results_leaky}
     \centering
     \begin{tabular}{l l l}
          \toprule
          Modelo & Épocas & RMSE \\
          \midrule
          LeNet & $12$ & $41.55$ \\
          AlexNet & $6$ & $14.38$\\
          \bottomrule
     \end{tabular}
\end{table}

Observa-se que houve uma resposta positiva da AlexNet que melhorou a qualidade das previsões para o problema considerado. Porém, observa-se uma tendência desta rede em prever valores médios, o que ainda enseja melhorias. Assim, ainda é necessário investigar outros parâmetros e modelos para o problema em questão.
