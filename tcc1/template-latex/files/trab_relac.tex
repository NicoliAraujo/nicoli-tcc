%!TEX root = ../sbc-template.tex
A proposta apresentada está relacionada com inúmeros trabalhos envolvendo a aplicação de redes neurais convolucionais e outros modelos de \emph{machine learning} para a estimação de idade de indivíduos.

Segundo \cite{fu2010age}, a idade pode ser inferida a partir de padrões distintos que emergem através da aparência da face. Técnicas comuns para a estimação da idade envolvem a dedução de modelos matemáticos a partir do estudo do crescimento de medidas da face e do crânio \cite{kwon1999age}, da textura do rosto \cite{lanitis2002toward}, da captura de tendências de envelhecimento a partir de várias imagens de indivíduos de mesma idade \cite{fu2007estimating} e a extração de características específicas relacionadas à idade \cite{suo2008design, lou2018expression}. Modelos de \emph{machine learning} também são utilizados para a tarefa, em especial as redes neurais artificiais, K-vizinhos mais próximos e máquinas de vetores de suporte.

Recentemente, a aplicação de redes neurais convolucionais em problemas de classificação e detecção de objetos em imagens têm obtido resultados significativamente positivos. Em \cite{vggnet, resnet, inception, redmon2016you, ssd} dentre outros trabalhos, são descritas arquiteturas robustas capazes de detectar dezenas de objetos em várias situações. Treinadas com conjuntos de dados visuais que contam com milhares de exemplos como a ImageNet \cite{ImagenetChall}, Pascal VOC \cite{pascalvoc} e COCO \cite{coco}, estas redes são conhecidas por seu bom desempelho. Algumas destas redes foram afinadas utilizando conjuntos de dados menores e especializados para a tarefa de estimação de idade.

O trabalho de Rothe em \cite{rothe2015dex} relata um método para estimação de idade aparente em imagens de faces imóveis utilizando \emph{deep learning}. O método proposto consiste de detectar uma face em uma imagem, para em seguida estimar sua idade. Para esta última tarefa, Propõe-se um conjunto de $20$ redes neurais convolucionais classificadoras com arquiteturas VGG-16 pré-treinadas com a base de dados visuais ImageNet, e ajustadas utilizando imagens disponibilizadas pelos sites do IMDb, da Wikipedia, e o conjunto de dados \emph{Looking At People}--LAP para anotação de idade aparente. Cada modelo tem como saída um número discreto entre $0$ e $100$, representando a idade prevista. A saída final do modelo consiste na localização do rosto e na média entre as idades previstas pelos $20$ redes para o rosto detectado. A solução atingiu um MAE (\emph{Mean Average Error}) de $3.221$ na fase de testes.

Em \cite{liu2015agenet} cria-se um estimador de idade composto pela fusão de um modelo regressor e outro classificador. Realiza-se um pré-processamento das imagens de entrada, que envolve a detecção das faces presentes em cada imagem, seguida pela etapa de localização de pontos de referência, como olhos, nariz e boca, e por fim há a normalização das faces. Dois métodos de normalização de face são testados, a normalização exterior e interior. Após este pré-processamento, as imagens resultantes são alimentadas a modelos de redes neurais convolucionais profundas inspiradas na \emph{GoogLeNet} \cite{inception}. O modelo sofreu modificações em sua arquitetura, como adição de normalização do \emph{batch}, remoção de camadas de \emph{dropout} e perda. Foram treinados e testados diversos modelos com variações no tipo de normalização da face, tamanho do corte dos rostos, tipo de tarefa preditiva, etc. Os modelos resultantes destas variações foram unidos em um conjunto, que conseguiu prever idades com MAE de $3.3345$.

Ademais, é possível encontrar resultados satisfatórios para a tarefa de aprendizado proposta utilizando modelos menos complexos. Com o objetivo de consolidar um método de classificação de idade e gênero, \cite{levi2015age} propõe uma rede neural convolucional de natureza mais simples, se comparada com as citadas acima. Sua arquitetura consiste em três camadas convolucionais com \emph{dropout} e funções de ativação \emph{ReLU}, seguidas por três camadas totalmente conectadas. A camada de saída tem como função de ativação a  \emph{Softmax}. A escolha por um design de rede menor é motivado pelo desejo de reduzir o risco de \emph{overfitting} e pela natureza do problema, que contém somente 8 classes de idade. O modelo é treinado utilizando apenas o conjunto de referência \emph{Adience}, composto por imagens não filtradas para classificação de idade e gênero. Considerando uma margem de erro de uma classe de idade vizinha, a melhor rede obteve acurácia de $84.7\% \pm 2.2$ ao empregar a técnica de sobre-amostragem.
