%!TEX root = ../sbc-template.tex

As \emph{Smart} TVs são o resultado da evolução tecnológica junto aos aparelhos de televisão domésticos. Possuem capacidades interativas ligadas à internet, acesso a conteúdo online, \emph{e-commerce} de conteúdo televisivo, navegação web e acesso a redes sociais. Estes aparelhos podem ser equipados com câmeras e microfones embutidos e transmitem conteúdo 2D ou até mesmo 3D. Neste último caso, em particular, os telespectadores fazem uso de óculos especiais \cite{samsung:smarttv}, \cite{perakakis2015proposed}.

Segundo a Pesquisa Nacional por Amostra de Domicílios realizada pelo IBGE em 2015, foi observado um total de 103 milhões de aparelhos de televisões em residências e pontos comerciais, das quais 16 milhões são de \emph{Smart} TVs. A pesquisa detalha que 94\% destas \emph{Smart} TVs foram adquiridas entre 2014 e 2015. Os números mostram um posterior aumento nas vendas de aparelhos televisores deste tipo, representando 68,2\% do total de televisores vendidos no primeiro semestre de 2017 \cite{pnad2015}.

Este aumento de vendas tem várias causas. Destacam-se os muitos benefícios resultantes do uso de \emph{Smart} TVs quando comparadas aos aparelhos convencionais \cite{shin2013smart}, \cite{differencebetween}. Em especial, cita-se o aumento da qualidade na transmissão, a utilização de aplicativos diversos e a possibilidade de acesso à conteúdo \emph{online} e \emph{on demand}, gratuitos ou mediante assinaturas. Além destes benefícios, cuja maioria é resultante da conectividade com a internet, outros fatores têm justificado o aumento das vendas e do interesse do público consumidor pelas \emph{Smart} TVs, tais como o encerramento da transmissão de sinal analógico da televisão aberta, a Copa do Mundo 2018 e a tecnologia 4K \cite{leiajabuscasmart,correiopnad,estadao:explosaovideosonline}.

Apesar da grande disponibilidade de conteúdo nas \emph{Smart} TVs e sua grande difusão nos lares brasileiros, é imprescindível levar em conta as restrições e recomendações deste conteúdo para o público alvo a que se destina. Tendo isto em mente, o processo de classificação indicativa integra o sistema de garantias dos direitos da criança e do adolescente quanto a promover, defender e garantir o acesso a espetáculos e diversões públicas adequados à condição de seu desenvolvimento, mas reserva-se o direito final aos pais e responsáveis quanto à escolha do conteúdo adequado a estes \cite{eca}. No Brasil, a \emph{Coordenação de Classificação Indicativa} (Cocind), vinculada ao Ministério da Justiça, é o órgão responsável pela classificação indicativa de obras destinadas à televisão e outros meios, incluindo até mesmo aplicativos.

Assim, buscando facilitar a interação entre telespectador e \emph{Smart} TV, um estimador de idade pode ser utilizado para facilitar a coleta de informações que contribuam para melhor experiência de provimento de conteúdo e de configurações personalizadas, e pode ser especialmente útil para a implementação de um controle parental mais eficiente, protegendo crianças e adolescentes de conteúdos inadequados à sua faixa etária.
