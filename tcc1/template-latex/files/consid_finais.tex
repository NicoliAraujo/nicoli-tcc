%!TEX root = ../sbc-template.tex
O objetivo deste trabalho consiste em elaborar estratégias inteligentes para estimação de idade de telespectadores de  \emph{Smart} TVs a partir de suas respectivas fotografias faciais. Para este fim, foram propostos, treinados e testados em caráter preliminar dois modelos de CNNs já bem estabelecidos na literatura, a LeNet e AlexNet, com dois perfis de hiperparâmetros cada um.

Com isto, observou-se uma melhora significativa na performance da AlexNet, enquanto o RMSE da LeNet não sofreu grandes mudanças. Quanto às saídas das redes, a LeNet exibiu valores positivos e negativos próximos de zero, e a AlexNet forneceu previsões que giravam em torno da média dos dados. Estes resultados são preliminares e certamente outros modelos e parâmetros serão investigados conforme previsto na metodologia e cronograma deste trabalho de conclusão de curso.

Nos próximos meses, os esforços estarão concentrados em pesquisar e adotar estratégias que possam minimizar os problemas identificados, como substituir as funções de ativação das camadas ocultas por outras variantes da \emph{ReLU}, adotar métodos específicos de inicialização de pesos, normalização de \emph{batch}, entre outros. Planeja-se também a proposição, o treinamento e teste de outras redes inspiradas em outros modelos canônicos.

O problema em questão é importante do ponto de vista prático para o desenvolvimento de diversas soluções de recomendação de conteúdo e controle parental em \emph{Smart} TVs, auxiliando no desenvolvimento destas soluções tecnológicas. Considerando a formação de uma bacharela em Engenharia de Computação, endereçar este problema permite a prática de diversos conceitos vistos ao longo do curso, em especial relacionados às disciplinas de Inteligência Artificial, Redes Neurais, Processamento Digital de Imagens, \emph{Machine Learning} e Sinais e Sistemas.
