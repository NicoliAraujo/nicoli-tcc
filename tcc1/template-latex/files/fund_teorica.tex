%!TEX root = ../sbc-template.tex
\subsection{Sistema Brasileiro de Classificação Indicativa}

O processo de classificação indicativa integra o sistema de garantias dos direitos da criança e do adolescente quanto a promover, defender e garantir o acesso a espetáculos e diversões públicas adequados à condição de seu desenvolvimento. Aos pais e responsáveis reserva-se o direito de escolher os conteúdos adequados aos seus pupilos, com possibilidade de bloqueio de acesso a programas ou canais de televisão, jogos eletrônicos e aplicativos ou autorização de acesso a diversões e espetáculos públicos, ainda que a classificação indique faixa etária superior à da criança ou do adolescente.

A Coordenação de Classificação Indicativa (Cocind) é o órgão responsãvel pela classificação indicativa de obras destinadas à televisão, ao cinema, à distribuição de vídeo doméstico, jogos eletrônicos, RPGs e aplicativos. A Cocind é subordinada ao Departamento de Políticas de Justiça, Classificação, Títulos e Qualificação (DPJUS), vinculado à Secretaria Nacional de Justiça do Ministério da Justiça.


Assim, a análise da classificação indicativa considera o grau de incidência de conteúdos de sexo e nudez, violência e drogas nas obras a serem avaliadas. O processo envolve a análise do conteúdo das obras a serem classificadas, a atribuição de classificação indicativa, verificação do cumprimento das normas de classificação indicativa, advertência por descumprimento das normas de classificação indicativa e comunicação aos órgãos compententes do descumprimento das normas de classificação indicativa.

O procedimento de atribuição da classificação indicativa é composto pela análise da documentação de instrução do processo, que envolve a apresentação ao Dejus da documentação e da obra a ser classificada; análise da obra, que deve conter a descrição dos conteúdos e a avaliação contextual; e pela atribuição da classificação indicativa como resultado da ponderação das fases descritiva e contextual.

As obras podem ser classificadas nas categorias mostradas na Tabela \ref{tab:categorias}.
\begin{table}[h!]
	\centering
	\begin{tabular}{p{3cm} p{2cm} p{9cm}}
		\hline
		Categoria & Símbolo & Descrição do Conteúdo \\
		\hline
		Livre & \includegraphics[width=0.05\textwidth]{img/livre.png} &
				Conteúdo predominantemente positivos ou que contenham imagens de violência fantasiosa, armas sem violência, mortes sem violência, ossadas e esqueletos sem violência, nudez não erótica e consumo moderado ou inusitado de drogas lícitas. \\
		\hline
		Não recomendado para menores de dez anos & \includegraphics[width=0.05\textwidth]{img/10anos.png} &
		 		Presença de armas com violência; medo ou tensão; angústia; ossadas e esqueletos com resquícios de ato de violênncia; atos criminosos sme violência; linguagem depreciativa; conteúdos educativos sobre sexo; descrições verbais do consumo de drogas lícitas; discussão sobre o tráfico de drogas; e o uso medicinal de drogas ilícitas.\\
		\hline
		Não recomendado para menores de doze anos & \includegraphics[width=0.05\textwidth]{img/12anos.png} &
				Ato violento; lesão corporal; descrição de violência; presença de sangue; sofrimento da vítima; morte natural ou acidental com violência; ato violento contra animais; exposição ao perigo; exposição de pessoas em situações constrangedoras ou degradantes; agressão verbal; obscenidade; bullying; exposição de cadáver; assédio sexual; supervalorização de beleza física; supervalorização do consumo; nudez velada; insinuação sexual; carícias sexuais; masturbação não explícita; linguagem chula; linguagem de conteúdo sexual; simulações de sexo; apelo sexual; consumo de drogas lícitas; indução ao uso de drogas lícitas; consumo irregular de medicamentos; menção a drogas ilícitas.\\
		\hline
		Não recomendado para menores de catorze anos & \includegraphics[width=0.05\textwidth]{img/14anos.png} &
				Morte intencional; estigma ou preconceito; nudez; erotização; vulgaridade; relação sexual não explícita; prostituição; insinuação do consumo de drogas ilícitas; descrições verbais do 	consumo de drogas ilícitas; e discussão sobre a descriminalização de drogas ilícitas.\\
		\hline
		Não recomendado para menores de dezesseis anos & \includegraphics[width=0.05\textwidth]{img/16anos.png} &
				Estupro; exploração sexual; coação sexual; tortura; mutilação; suicídio; violência gratuita ou banalização da violênciaç aborto, pena de morte ou eutanásia; relação sexual intensa não explícita; produção ou tráfico de qualquer droga ilícita, consumo de drogas ilícitas; indução ao consumo de drogas ilícitas.\\
		\hline
		Não recomendado para menores de dezoito anos & \includegraphics[width=0.05\textwidth]{img/18anos.png} &
				Violência de forte impacto; elogio; glamourização e/ou apologia à violência; crueldade; crimes de ódio; pedofilia; sexo explícito; situações sexuais complexas ou de forte impacto; apologia ao uso de drogas ilícitas.\\
		\hline
	\end{tabular}
	\caption{Faixas de Classificação Indicativa}
	\label{tab:categorias}
\end{table}


Obras audiovisuais a serem transmitidas em televisão aberta sem classificação indicativa anterior não necessitam passar por análise prévia desde que o interessado requeira a autoclassificação munido do documento de ficha técnica de classificação (disponível no site do MJ); da cópia do pagamento da contribuição para o desenvolvimento da indústria conematográfica nacional, se necessário, ou do formulário do Dejus de insenção de pagamento; e sinopse detalhada, no caso de obras de ficção.

Obras audiovisuais a serem transmitidas em televisão por assinatura devem apresentar os simbolos e demais informações de classificação indicativa CONFORME O GUIA. Deve também disponibilizar sistema de bloqueio de canais ou programas; digulgar objetiva e amplamente a forma de utilização dos sistemas de controle e de bloqueio; e possiblite ao usuário acessar a qualquer tempo a informação completa de sua classificação indicativa.
