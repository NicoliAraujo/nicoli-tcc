\emph{Machine Learning} (ML), também chamado de Aprendizado de Máquina, é uma subárea da Inteligência Artificial que trata da criação de modelos que modificam ou adaptam suas ações para que elas se tornem mais acuradas. A acurácia neste domínio é entendida como uma medida do quão bem as escolhas realizadas pelos modelos refletem escolhas corretas. \todo{Citação?}

De maneira análoga aos humanos e outros animais, os algoritmos de ML precisam aprender. Este aprendizado, também chamado de aquisição de experiência, é realizado a partir de dados que refletem padrões históricos do problema que está sendo considerado.

\todo{O texto está muito superficial. Precisa ser melhorado antes de uma revisão detalhada...}

 Estes, ao se depararem com determinada, costumam tentar lembrar-se se da última vez em que estiveram em uma situação parecida, tentaram alguma ação que pode ter dado certo -- então deve ser repetida-- ou errado -- então deve tentar algo diferente --adaptação \cite{marsland2015machine,goodfellow2016deep},


  \cite{}. De acordo com a definição clássica de \cite{mitchell1997machine}, um algoritmo que aprende a partir da experiência $E$ quanto a um conjunto de tarefas $T$ e medida de performance $P$, se sua performance nas tarefas em $T$, medida por $P$, melhora com a experiência $E$.
Algumas tarefas que podem ser atacadas utilizando aprensdizado de máquina são a classificação, regressão, transcrição, tradução automática, detecção de anomalia, síntese e amostragem \cite{goodfellow2016deep}.

\todo{gancho?}
