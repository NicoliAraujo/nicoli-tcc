%!TEX root = ../sbc-template.tex
As \emph{Smart} TVs são o resultado da evolução tecnológica junto aos aparelhos de televisão domésticos. Possuem capacidades interativas ligadas à internet, acesso a conteúdo online, \emph{e-commerce} de conteúdo televisivo, navegação web e acesso a redes sociais. Estes aparelhos podem ser equipados com câmeras e microfones embutidos e  são aptos a transmitirem conteúdo 2D ou até mesmo 3D especiais \cite{samsung:smarttv,perakakis2015proposed}.

Segundo a Pesquisa Nacional por Amostra de Domicílios realizada pelo IBGE em 2015, foi observado um total de $103$ milhões de aparelhos de televisões em residências e pontos comerciais, das quais $16$ milhões são de \emph{Smart} TVs. A pesquisa detalha que $94\%$ destas \emph{Smart} TVs foram adquiridas entre $2014$ e $2015$. Os números mostram um posterior aumento nas vendas de aparelhos televisores deste tipo, representando $68,2\%$ do total de televisores vendidos no primeiro semestre de $2017$ \cite{pnad2015}.

Este aumento de vendas tem várias causas, das quais destacam-se os muitos benefícios resultantes do uso de \emph{Smart} TVs quando comparadas aos aparelhos convencionais \cite{shin2013smart,differencebetween}. Em especial, cita-se o aumento da qualidade na transmissão, a utilização de aplicativos diversos e a possibilidade de acesso à conteúdo \emph{online} e \emph{on demand}, gratuitos ou mediante assinaturas. Além destes benefícios, cuja maioria é resultante da conectividade com a internet, outros fatores têm justificado o aumento das vendas e do interesse do público consumidor pelas \emph{Smart} TVs, tais como o encerramento da transmissão de sinal analógico da televisão aberta, a Copa do Mundo 2018 e a tecnologia 4K \cite{leiajabuscasmart,correiopnad,estadao:explosaovideosonline}.

Considerando a grande difusão das \emph{Smart} TVs nos lares brasileiros, é essencial que estes aparelhos sejam capazes de capturar o perfil e o interesse dos seus telespectadores a fim de oferecer uma experiência mais rica. A recomendação de conteúdo, por exemplo, pode levar em conta características individuais, tais como idade e sexo. Porém, se fornecidos de maneira habitual, via preenchimento de formulários, além de ser uma tarefa massante, pode não refletir de maneira realística o perfil individual dos vários usuários que podem estar à frente de uma \emph{Smart} TV em um determinado momento.

Apesar das dificuldades práticas mencionadas, é interessante notar que muitas \emph{Smart} TVs possuem dispositivos para captura de imagens, como câmeras, pois também costumam dispor de aplicações para troca de mensagens de vídeo \cite{Guardian:CameraSmartv}. Respeitadas as preferências de privacidade de cada usuário, se estas câmeras forem habilidades para aquisição de imagens daqueles que estão à frente do televisor, então é possível usá-las como entrada para sistemas inteligentes de identificação de características, cujas previsões podem ser usadas, por exemplo, para recomendação de conteúdo. No caso da idade, em particular, é possível usar estas informações para realizar um controle parental mais eficiente, protegendo crianças e adolescentes de conteúdos inadequados à sua faixa etária.

Diante do que foi exposto, esta proposta de trabalho de conclusão de curso considera o desenvolvimento de estratégias inteligentes, baseadas na utilização de técnicas de \emph{Deep Learning}, para estimação da idade de telespectadores a partir de fotografias faciais. Embora a estimação de outras características também pudesse ser realizada mediante a análise de fotografias faciais, desde gênero até a presença de doenças, optou-se pela idade por ser um atributo comum a todos os telespectadores, pelo potencial de aplicações, pela existência de bases de dados adequadamente rotuladas com este atributo e pelo menor potencial de infringência das searas privadas dos usuários.

\subsection{Objetivos}
%!TEX root = ../sbc-template.tex
O objetivo geral deste trabalho consiste em propor um estimador de idade para telespectadores de \emph{Smart} TVs. Para alcançar esta meta, alguns objetivos específicos precisam ser contemplados, a citar:

\begin{itemize}
     \item Formular um referencial teórico sobre redes neurais convolucionais, modelo de \emph{machine learning} considerado, contemplando suas características, principais arquiteturas, métodos de treinamento e teste;
     \item Consolidar uma base de dados para a tarefa de \emph{machine learning} proposta, contemplando exemplos realísticos;
     \item Identificar tecnologias adequadas para implementar o estimador proposto;
     \item Propor, treinar e testar diferentes arquitteturas  de redes neurais convolucionais para a tarefa em questão;
     \item Avaliar comparativamente os estimadores propostos.
\end{itemize}


\subsection{Justificativa}
%!TEX root = ../../novoIndex.tex
A realização de um trabalho de conclusão de curso desta natureza é justificada por várias razões. No contexto da interação entre telespectador e \emph{Smart} TV, um estimador de idade pode ser utilizado para facilitar a coleta de informações que contribuam para melhor experiência de provimento de conteúdo e de configurações personalizadas. Em particular, a estimação de idade dos telespectadores pode ser especialmente empregada na implementação de um controle parental mais eficiente, protegendo crianças e adolescentes de conteúdos inadequados à sua faixa etária.

Um outro aspecto que ressalta a importância da realização de um trabalho desta natureza é a prática e a proposição de soluções envolvendo \emph{Machine Learning}. Esta é uma área de vanguarda na Computação e seu potencial para resolução de problemas práticos está em franco desenvolvimento. Ao considerar a elaboração do estimador proposto, será necessário dominar conhecimentos de ferramental tecnológico atual, o que pode colaborar na minimização da distância entre o profissional em formação e os anseios do mercado de trabalho da área.

Por fim, há que se mencionar a relação entre a área de pesquisa considerada neste trabalho de conclusão de curso e o Laboratório de Sistemas Inteligentes (LSI). Este trabalho alinha-se com os objetivos desta iniciativa do Núcleo de Computação (NUCOMP), motivando o desenvolvimento de uma solução inovadora que utiliza técnicas da Inteligência Artificial.


\subsection{Metodologia}\label{sec:metodo}
%!TEX root = ../../novoIndex.tex
A metodologia para o desenvolvimento deste trabalho consistiu na realização da \emph{fundamentação teórica sobre Machine Learning}, em especial contemplando os conceitos relativos às redes neurais convolucionais. Para tanto, considerou-se a literatura desta área para que haja o entendimento das bases matemáticas deste modelo computacional, como funcionam, quais as características e as arquiteturras mais importantes. Neste estudo, além dos aspectos teóricos, foram considerados os ambientes de desenvolvimento, bibliotecas e outras tecnologias para implementação dos conceitos contemplados.

Os demais passos que compõem a metodologia deste trabalho baseiam-se no \emph{fluxo de atividades de machine learning} \cite{marsland2015machine}. Inicialmente, houve a aquisição e o pré-processamento de imagens para \emph{consolidar uma base de dados} para esta tarefa de aprendizado. Nesta etapa, foi considerada a literatura e uma base de dados já disponível e apropriadamente anotada, com licença livre de utilização.

A seguir, houve a \emph{proposição de diferentes modelos de redes neurais convolucionais} para a tarefa de aprendizado considerada. Nesta etapa, foram elencados diferentes parâmetros e hiperparâmetros de configuração, bem como arquiteturas. Estes procedimentos visaram consolidar um espaço de busca de modelos que possam endereçar a tarefa de maneira mais eficiente.

O próximo estágio consistiu no \emph{treinamento das redes neurais convolucionais} para o problema em questão. Durante este processo, uma parte da base de dados foi apresentada aos modelos para que houvesse o ajuste de pesos, compreendendo o aprendizado das características relevantes. O treinamento das redes ocorreu utilizando computação em nuvem e computadores disponíveis no Laboratório de Sistemas Inteligentes (LSI), tendo em vista a infra-estrutura de hardware necessária para realizar este procedimento.

Seguiu-se então o \emph{teste das redes}, respeitando uma abordagem de validação cruzada e utilizando métricas de desempenho apropriadas. O objetivo desta fase consistiu em aferir os modelos propostos e treinados quanto à sua capacidade de generalização.

Por fim, para identificação de um modelo mais adequado à esta tarefa, as \emph{métricas de desempenho foram comparadas} e os melhores modelos elencados a partir destes valores, apontando assim um estimador apropriado para o problema inicialmente considerado.

Alem destas atividades, há que se considerar a escrita da proposta e do projeto final do trabalho de conclusão de curso, bem como as defesas parcial e final.


\subsection{Cronograma}
%!TEX root = ../sbc-template.tex

O cronograma de realização das atividades pode ser visto na Tabela \ref{tab:cronograma}. As atividades listadas possuem relação com a metodologia detalhada na seção anterior, compreendendo os requisitos elementares para a realização deste trabalho.
\newline

\begin{table}{H}
\scalefont{0.8}
\caption{Cronograma de atividades levando em consideração os dez meses (de $02/2018$ a $12/2018$) para a realização do TCC.}
\label{tab:cronograma}

\begin{center}
\begin{small}
\begin{tabular}{p{5cm}cccccccccccc}
  \toprule
  & &  &  & &  & \textbf{2018}  & &  &  &  &  & \\
                                        & \textbf{02} & \textbf{03} & \textbf{04} & \textbf{05} & \textbf{06} & \textbf{07} & \textbf{08} & \textbf{09} & \textbf{10} & \textbf{11} & \textbf{12} \\
  \midrule
  \textbf{Escrita da Proposta}          &      X      &      X      &      X      &      X      &      X      &             &             &             &             &             &             \\
  \textbf{Fundamentação Teórica sobre
  Machine Learning}                     &      X      &      X      &      X      &      X      &             &             &             &             &             &             &             \\
  \textbf{Consolidação da Base de Dados}&             &      X      &      X      &             &             &             &             &             &             &             &             \\
  \textbf{Proposição de Modelos de
  Redes Neurais Convolucionais}         &             &             &             &      X      &      X      &      X      &      X      &      X      &             &             &             \\
  \textbf{Defesa da Proposta}          &             &             &             &             &      X      &             &             &             &             &             &             \\
  \textbf{Escrita do Trabalho Final}    &             &             &             &             &             &      X      &      X      &      X      &      X      &      X      &      X      \\
  \textbf{Treinamento das
  Redes Neurais Convolucionais}         &             &             &             &             &      X      &      X      &      X      &      X      &      X      &      X       &            \\
  \textbf{Teste das Redes
  Neurais Convolucionais}               &             &             &             &             &      X      &      X      &      X      &      X      &      X      &      X       &     X      \\
  \textbf{Comparação de Metricas
  de Desempenho}                        &             &             &             &             &             &      X      &      X      &      X      &      X      &      X      &      X      \\
  \textbf{Defesa do Trabalho Final}     &             &             &             &             &             &             &             &             &             &             &      X      \\
  \bottomrule
\end{tabular}
\end{small}
\end{center}
\end{table}


\subsection{Organização do Documento}
\todo{Falta!}
