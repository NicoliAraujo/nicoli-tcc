%!TEX root = ../sbc-template.tex

\subsection{Tarefa de Previsão Considerada}

% Classificação multi-classe


\subsection{Elaboração e Descrição da Base de Dados}



\subsection{Modelos de CNN Considerados}
\subsection{Parâmetros e Hiperparâmetros}

\subsection{Métricas de Desempenho}

% Escrever sobre micro-averaging
\subsubsection{Micro-Média}
\todo{como traduzir?}
A micro-média é uma medida de performance utilizada em problemas de classificação multi-classe em que não há balanceamento entre os elementos de cada tipo. Assim, cada classe recebe um peso de acordo com sua frequência entre os exemplos, o que faz com que o $F1$-score das classes com mais exemplos infuencie na métrica mais que o de classes com menos exemplos \cite{ghamrawi2005collective}. A performance é dada pela média de todos os exemplos como se pertencessem a uma mesma classe, como mostra a Equação \ref{eq:micro_averaging}, sendo $PR$, $Re$ e $F1$ a precisão, revocação e $F1$-score, respectivamente, e TP, FP, FN, positivo verdadeiro, positivo falso e negativo verdadeiro\cite{kubat2016introduction}.
\begin{align}\label{eq:micro_averaging}
	\begin{split}
		Pr^{\mu} = \frac{\sum_{i=1}^L TP_i}{\sum_{i=1}^L TP_i + FP_i}
		\\ \\
		Re^{\mu} = \frac{\sum_{i=1}^L TP_i}{\sum_{i=1}^L TP_i + FN_i}
		\\ \\
		F1^{\mu} = \frac{2 \times Pr^{\mu} \times Re^{\mu}}{Pr^{\mu} + Re^{\mu}}
	\end{split}
\end{align}

\subsection{Etapa de Treinamento}

\subsection{Etapa de Testes}
