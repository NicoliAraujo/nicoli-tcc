%!TEX root = ../sbc-template.tex

\subsection{Tarefa de Previsão Considerada}
A tarefa de previsão considerada é a regressão. O modelo tem como entrada um rosto, e fornece como saída uma idade prevista para aquele rosto.

\subsection{Elaboração e Descrição da Base de Dados}
%!TEX root = ../sbc-template.tex
Para a tarefa de aprendizado apresentada, dispôs-se da base de dados experimentais IMDb, composto de 452132 exemplares contendo imagens e outras informações de $20,284$ dos atores mais populares listados no site IMDb. O conjunto de dados foi construído utilizando técnicas de \emph{web crawling} aplicadas aos perfis de atores do IMDb, em que foram coletadas todas as imagens relacionadas à celebridade, além de informações como data de nascimento, nome e gênero. Assim sendo, cada imagem está relacionada às informações de uma celebridade.

A base de dados IMDb foi utilizada em \cite{rothe2015dex} para o treinamento de um detector de faces e estimador de idade relatado na Seção \ref{sec:trab_relac}. Com isto, há também as coordenadas da localização de um rosto detectado na imagem, além de uma pontuação atribuída ao rosto pelo detector referentes ao grau de certeza de que há um rosto naquela imagem. Partindo da possibilidade de haver mais de um rosto por imagem, uma segunda pontuação é atribuída pelo detector, referente ao grau de certeza de que há outro rosto na mesma imagem.

Neste contexto, cada item desta base de dados é referente a uma imagem, cujas informações estão descritas nos atributos do conjunto de dados, que compreendem o nome, gênero, data de nascimento e um número de identificação da celebridade cujo perfil estava atrelado à imagem, o endereço da foto em disco, a suposta localização da face da celebridade, e pontuações referentes a duas possíveis faces encontradas. Assim, há exemplos de imagens em que há apenas um rosto, como mostrado na Tabela \ref{tab:um_deniro}. Já na Tabela \ref{tab:dois_deniro_correto} está o exemplo de uma imagem onde há mais de um rosto, porém a localização do rosto está correta. Por fim, na Tabela \ref{tab:dois_deniro_errado} há uma imagem com mais de um rosto, porém o rosto identificado neste item não é o da celebridade cujos dados estão referenciados.

\begin{table}[ht]
     \caption{Exemplo de imagem do conjunto de dados contendo apenas um rosto.}
     \label{tab:um_deniro}
          \begin{minipage}[c]{0.62\linewidth}
          \begin{small}
          \centering
          \begin{tabular}{p{3.3cm} p{5cm}}\hline
               Meta-dado & Valor \\ \hline
               ID Celebridade & 16349 \\
               Nome & Robert De Niro \\
               Endereço da imagem & \footnotesize{imdb$/$34$/$nm0000134$\_$rm334009 0368$\_$1943-8-17$\_$2011.jpg} \\
               Pontuação da Face & $5.21396$ \\
               Pontuação da Segunda Face & NaN \\
               Localização da Face & $(663.65, 992.475, $ $590.134, 918.959)$ \\
               Data de Nascimento  & $1943-08-17$\\
               Ano da Foto & 2011 \\
               Gênero & Masculino \\ \hline
          \end{tabular}
     \end{small}
     \end{minipage}
     \hfill
     \begin{minipage}[c]{0.35\linewidth}
          \centering
          \includegraphics[width=\linewidth]{img/deniro_plt}
     \end{minipage}
\end{table}

\begin{table}[ht]
     \caption{Exemplo de imagem do conjunto de dados contendo mais de um rosto com a classificação correta.}
     \label{tab:dois_deniro_correto}
          \begin{minipage}[c]{0.62\linewidth}
          \begin{small}
          \centering
          \begin{tabular}{p{3.3cm} p{5cm}}\hline
               Meta-dado & Valor \\ \hline
               ID Celebridade & 16349 \\
               Nome & Robert De Niro \\
               Endereço da imagem & \footnotesize{imdb$/$34$/$nm0000134$\_$rm17663 60064$\_$1943-8-17$\_$2010.jpg} \\
               Pontuação da Face & $5.12527$ \\
               Pontuação da Segunda Face & $5.08887$ \\
               Localização da Face & $(914.886, 1426.31, $ $287.31, 798.734)$ \\
               Data de Nascimento  & $1943-08-17$\\
               Ano da Foto & 2010 \\
               Gênero & Masculino \\ \hline
          \end{tabular}
     \end{small}
     \end{minipage}
     \hfill
     \begin{minipage}[c]{0.35\linewidth}
          \centering
          \includegraphics[width=\linewidth]{img/deniro_many_plt_correto}
     \end{minipage}
\end{table}

\begin{table}[ht]
     \caption{Exemplo de imagem do conjunto de dados contendo mais de um rosto com a classificação errônea.}
     \label{tab:dois_deniro_errado}
          \begin{minipage}[c]{0.62\linewidth}
          \begin{small}
          \centering
          \begin{tabular}{p{3.3cm} p{5cm}}\hline
               Meta-dado & Valor \\ \hline
               ID Celebridade & 16349 \\
               Nome & Robert De Niro \\
               Endereço da imagem & \footnotesize{imdb$/$34$/$nm0000134$\_$rm14800 44288$\_$1943-8-17$\_$2012.jpg} \\
               Pontuação da Face & $5.51656$ \\
               Pontuação da Segunda Face & $4.55379$ \\
               Localização da Face & $(1392.72, 1614.18, $ $225.55, 447.003)$ \\
               Data de Nascimento  & $1943-08-17$\\
               Ano da Foto & 2012 \\
               Gênero & Masculino \\ \hline
          \end{tabular}
     \end{small}
     \end{minipage}
     \hfill
     \begin{minipage}[c]{0.35\linewidth}
          \centering
          \includegraphics[width=\linewidth]{img/deniro_many_plt_errado}
     \end{minipage}
\end{table}

A versão original das imagens do conjunto de dados IMDb ocupava 267 GB em disco. Porém, uma versão pré-processada dessas imagens está disponível, contendo as faces recortadas com $40\%$ da largura e altura da imagem original, totalizando $7,1$ GB de dados. Esta versão foi considerada neste trabalho.


\subsection{Modelos de CNN Considerados}
Considerou-se as arquiteturas LeNet e AlexNet. Todas as funções de ativação tangente hiperbólica foram substituídas pela ReLU. A implemetação da AlexNet seguiu a prática atual de utilizar apenas uma GPU no treinamento, portanto as camadas dividas foram unificadas. A normalização original que utilizava LBN foi substituída pela normalização em batch, que tem se mostrado mais efetiva \todo{citar chollet}.

\subsection{Parâmetros e Hiperparâmetros}

\subsection{Métricas de Desempenho}
A perda utilizada no treinamento foi o Erro Médio Quadrático (Mean Squared Error), e as métricas utilizadas foram o MSE e o MAE
% Escrever sobre micro-averaging
% \subsubsection{Micro-Média}
% \todo{como traduzir?}
% A micro-média é uma medida de performance utilizada em problemas de classificação multi-classe em que não há balanceamento entre os elementos de cada tipo. Assim, cada classe recebe um peso de acordo com sua frequência entre os exemplos, o que faz com que o $F1$-score das classes com mais exemplos infuencie na métrica mais que o de classes com menos exemplos \cite{ghamrawi2005collective}. A performance é dada pela média de todos os exemplos como se pertencessem a uma mesma classe, como mostra a Equação \ref{eq:micro_averaging}, sendo $PR$, $Re$ e $F1$ a precisão, revocação e $F1$-score, respectivamente, e TP, FP, FN, positivo verdadeiro, positivo falso e negativo verdadeiro\cite{kubat2016introduction}.
% \begin{align}\label{eq:micro_averaging}
% 	\begin{split}
% 		Pr^{\mu} = \frac{\sum_{i=1}^L TP_i}{\sum_{i=1}^L TP_i + FP_i}
% 		\\ \\
% 		Re^{\mu} = \frac{\sum_{i=1}^L TP_i}{\sum_{i=1}^L TP_i + FN_i}
% 		\\ \\
% 		F1^{\mu} = \frac{2 \times Pr^{\mu} \times Re^{\mu}}{Pr^{\mu} + Re^{\mu}}
% 	\end{split}
% \end{align}

\subsection{Etapa de Treinamento}

\subsection{Etapa de Testes}
