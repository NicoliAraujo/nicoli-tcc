%!TEX root = ../sbc-template.tex

\subsection{Tarefa de Aprendizado}
A tarefa de aprendizado considerada para a estimação de idade de telespectadores é a regresssão. Neste contexto, uma imagem em cores RGB de dimensões $224 \times 224$ pixels contendo uma face humana centralizada será fornecida como entrada. A saída desejada é a estimativa de idade, em anos, da pessoa correspondente, conforme exemplificado na Figura \ref{fig:deniro_cnn}. Esta tarefa será abordada segundo o paradigma de aprendizado supervisionado.

\begin{figure}
  \centering
     \caption{Tarefa de aprendizado}
     \includegraphics[width=\textwidth]{img/deniro_cnn}
     \label{fig:deniro_cnn}
\end{figure}

Os dados disponíveis para este contexto seráo particionados em três conjuntos disjuntos, sendo $70\%$ reservados para o treino, $10\%$ para validação e $20\%$ para teste. Esta partição obedece à tecnica \emph{Holdout} de validação cruzada \cite{brink2016real}.

Os modelos propostos para esta tarefa terão seu desempenho aferido perante os dados do conjunto de testes de acordo com a métrica de desempenho \emph{Root Mean Squared Error} (RMSE). Esta métrica considera a diferença entre cada um dos valores previstos $\hat{y}$ e os reais $y$, e posteriormente quantifica uma média imune à variação positiva ou negativa desta diferença. A Equação \ref{eq:rmse} denota o cálculo do RMSE \cite{brink2016real}.

\begin{equation}\label{eq:rmse}
     \textrm{RMSE} = \sqrt{\frac{1}{n} \sum_{i=1}^n (y_i - \hat{y})^2}.
\end{equation}

\subsection{Conjunto de Dados}
%!TEX root = ../sbc-template.tex
Para a tarefa de aprendizado apresentada, dispôs-se da base de dados experimentais IMDb, composto de 452132 exemplares contendo imagens e outras informações de $20,284$ dos atores mais populares listados no site IMDb. O conjunto de dados foi construído utilizando técnicas de \emph{web crawling} aplicadas aos perfis de atores do IMDb, em que foram coletadas todas as imagens relacionadas à celebridade, além de informações como data de nascimento, nome e gênero. Assim sendo, cada imagem está relacionada às informações de uma celebridade.

A base de dados IMDb foi utilizada em \cite{rothe2015dex} para o treinamento de um detector de faces e estimador de idade relatado na Seção \ref{sec:trab_relac}. Com isto, há também as coordenadas da localização de um rosto detectado na imagem, além de uma pontuação atribuída ao rosto pelo detector referentes ao grau de certeza de que há um rosto naquela imagem. Partindo da possibilidade de haver mais de um rosto por imagem, uma segunda pontuação é atribuída pelo detector, referente ao grau de certeza de que há outro rosto na mesma imagem.

Neste contexto, cada item desta base de dados é referente a uma imagem, cujas informações estão descritas nos atributos do conjunto de dados, que compreendem o nome, gênero, data de nascimento e um número de identificação da celebridade cujo perfil estava atrelado à imagem, o endereço da foto em disco, a suposta localização da face da celebridade, e pontuações referentes a duas possíveis faces encontradas. Assim, há exemplos de imagens em que há apenas um rosto, como mostrado na Tabela \ref{tab:um_deniro}. Já na Tabela \ref{tab:dois_deniro_correto} está o exemplo de uma imagem onde há mais de um rosto, porém a localização do rosto está correta. Por fim, na Tabela \ref{tab:dois_deniro_errado} há uma imagem com mais de um rosto, porém o rosto identificado neste item não é o da celebridade cujos dados estão referenciados.

\begin{table}[ht]
     \caption{Exemplo de imagem do conjunto de dados contendo apenas um rosto.}
     \label{tab:um_deniro}
          \begin{minipage}[c]{0.62\linewidth}
          \begin{small}
          \centering
          \begin{tabular}{p{3.3cm} p{5cm}}\hline
               Meta-dado & Valor \\ \hline
               ID Celebridade & 16349 \\
               Nome & Robert De Niro \\
               Endereço da imagem & \footnotesize{imdb$/$34$/$nm0000134$\_$rm334009 0368$\_$1943-8-17$\_$2011.jpg} \\
               Pontuação da Face & $5.21396$ \\
               Pontuação da Segunda Face & NaN \\
               Localização da Face & $(663.65, 992.475, $ $590.134, 918.959)$ \\
               Data de Nascimento  & $1943-08-17$\\
               Ano da Foto & 2011 \\
               Gênero & Masculino \\ \hline
          \end{tabular}
     \end{small}
     \end{minipage}
     \hfill
     \begin{minipage}[c]{0.35\linewidth}
          \centering
          \includegraphics[width=\linewidth]{img/deniro_plt}
     \end{minipage}
\end{table}

\begin{table}[ht]
     \caption{Exemplo de imagem do conjunto de dados contendo mais de um rosto com a classificação correta.}
     \label{tab:dois_deniro_correto}
          \begin{minipage}[c]{0.62\linewidth}
          \begin{small}
          \centering
          \begin{tabular}{p{3.3cm} p{5cm}}\hline
               Meta-dado & Valor \\ \hline
               ID Celebridade & 16349 \\
               Nome & Robert De Niro \\
               Endereço da imagem & \footnotesize{imdb$/$34$/$nm0000134$\_$rm17663 60064$\_$1943-8-17$\_$2010.jpg} \\
               Pontuação da Face & $5.12527$ \\
               Pontuação da Segunda Face & $5.08887$ \\
               Localização da Face & $(914.886, 1426.31, $ $287.31, 798.734)$ \\
               Data de Nascimento  & $1943-08-17$\\
               Ano da Foto & 2010 \\
               Gênero & Masculino \\ \hline
          \end{tabular}
     \end{small}
     \end{minipage}
     \hfill
     \begin{minipage}[c]{0.35\linewidth}
          \centering
          \includegraphics[width=\linewidth]{img/deniro_many_plt_correto}
     \end{minipage}
\end{table}

\begin{table}[ht]
     \caption{Exemplo de imagem do conjunto de dados contendo mais de um rosto com a classificação errônea.}
     \label{tab:dois_deniro_errado}
          \begin{minipage}[c]{0.62\linewidth}
          \begin{small}
          \centering
          \begin{tabular}{p{3.3cm} p{5cm}}\hline
               Meta-dado & Valor \\ \hline
               ID Celebridade & 16349 \\
               Nome & Robert De Niro \\
               Endereço da imagem & \footnotesize{imdb$/$34$/$nm0000134$\_$rm14800 44288$\_$1943-8-17$\_$2012.jpg} \\
               Pontuação da Face & $5.51656$ \\
               Pontuação da Segunda Face & $4.55379$ \\
               Localização da Face & $(1392.72, 1614.18, $ $225.55, 447.003)$ \\
               Data de Nascimento  & $1943-08-17$\\
               Ano da Foto & 2012 \\
               Gênero & Masculino \\ \hline
          \end{tabular}
     \end{small}
     \end{minipage}
     \hfill
     \begin{minipage}[c]{0.35\linewidth}
          \centering
          \includegraphics[width=\linewidth]{img/deniro_many_plt_errado}
     \end{minipage}
\end{table}

A versão original das imagens do conjunto de dados IMDb ocupava 267 GB em disco. Porém, uma versão pré-processada dessas imagens está disponível, contendo as faces recortadas com $40\%$ da largura e altura da imagem original, totalizando $7,1$ GB de dados. Esta versão foi considerada neste trabalho.


\subsection{Limpeza e Pré-processamento dos dados}

A fim de adequar melhor o conjunto de dados para os modelos de CNNs utilizados, realizou-se uma limpeza e pré-processamento dos meta-dados e das imagens da base IMDb, que se iniciou com o cálculo do atributo alvo, a idade, a partir dos atributos originais fornecidos. A idade foi aferida através da data de nascimento da celebridade e do ano em que a fotografia em questão foi capturada.

Uma análise do conjunto de dados revelou a presença de itens com idade e gênero apresentando valores nulos, inválidos ou negativos, que foram descartados. Observou-se também a presença de múltiplos exemplos referentes à mesma pessoa com a mesma idade. Houve a remoção de tais exemplos, a fim de evitar que a apresentação de um mesmo rosto com a mesma idade provocasse \emph{overfitting} nos modelos. Exemplos atípicos, possivelmente resultado de rotulação incorreta, como idade maior que $100$ anos ou não compatível com os dados da celebridade referida nos meta-dados também foram descartados. Os atributos de pontuação de rostos foram úteis para identificar e remover exemplos em que não havia nenhum rosto identificado, ou em que havia mais de uma face na imagem. Este descarte foi realizado com o objetivo de eliminar rotulações errôneas, como a mostrada na Tabela \ref{tab:deniro_many_plt_errado}.

A última etapa consistiu na padronização das dimensões das imagens. Considerando a literatura, definiu-se o tamanho para $224 \times 224$ \emph{pixels} e o modo RGB como padrões. Por fim, após a padronização das imagens de entrada, o cálculo do atributo alvo idade, a adequação do caminho para as imagens em disco e a remoção de exemplos impróprios, seguiu-se o descarte dos outros meta-dados irrelevantes para a tarefa de estimação de idade de um indivíduo a partir de imagem. A data em que a foto foi tirada, nome, número de identificação, gênero, data de nascimento, localização do rosto da celebridade e pontuações de rostos nas imagens foram removidos.

Por fim, o conjunto de dados consolidado consiste de $47.950$ exemplos contendo imagens e idades de $14.607$ celebridades distintas. O histograma de frequência da distribuição de idades de 0 a 100 anos presente nos exemplos da base de dados pode ser visualizado na Figura \ref{fig:hist}. Este total foi então dividido como proposto: conjunto de treinamento, contendo $70\%$ dos exemplos, ou seja, $33.565$ amostras; conjunto de validação, referente a $10\%$ dos dados, ou seja, $4.795$ itens; e, por fim, conjunto de testes, contendo os $20\%$ restantes, ou seja, $9.590$ exemplos.

\begin{figure}
    \centering
     \includegraphics[width=0.7\textwidth]{img/idade_hist_clean}
     \caption{Histograma de frequência da idade do conjunto de dados utilizado.}
     \label{fig:hist}
\end{figure}

\subsection{Modelos de CNN Considerados}

Levando em conta a adoção de CNNs como o modelo de Aprendizado de Máquina a ser usado neste trabalho, considerou-se a utilização das arquiteturas LeNet e AlexNet. Todas as funções de ativação tangente hiperbólica disponíveis na versão original destas redes foram substituídas pela função ReLU, por ser mais eficiente computacionalmente e por promover uma convergência mais rápida. A implementação da AlexNet seguiu a prática atual de utilizar apenas uma GPU no treinamento, portanto as camadas dividas foram unificadas \cite{tensorflow:alexnet}.

O número de épocas para treinamento foi igual a $100$, o \emph{batch size} igual a $64$, o método de otimização do gradiente descendente foi o \emph{Adam} e as taxas de treinamento adotadas foram obtidas de maneira experimental, observando a perda obtida a cada época.
