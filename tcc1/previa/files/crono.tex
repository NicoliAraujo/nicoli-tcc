%!TEX root = ../sbc-template.tex

Uma visão geral do cronograma de atividades deste trabalho de conclusão de curso pode ser vista na Tabela \todo{numero da tabela}. Elas possuem relação com a metodologia detalhada na seção \ref{sec:metodo}, que detalha as regras e diligências estabelecidas para realizar este trabalho.l
\newline

\begin{table}[ht]
%\caption{Cronograma de atividades levando em consideração os dez \todo{quantos?} meses (de $0l2/2018$ a $12/2018$) para a realização do TCC.}
\label{tab:cronograma}
\begin{center}
\begin{small}
\begin{tabular}{p{5cm}ccccccccccccc}
  \toprule
                & &  &  & &  &  & \textbf{2018} &  &  &  &  & & \\
                & \textbf{02} & \textbf{03} & \textbf{04} & \textbf{05} & \textbf{12} & \textbf{06} & \textbf{07} & \textbf{08} & \textbf{09} & \textbf{10} & \textbf{11} & \textbf{12}\\
  \midrule
  \textbf{Estudo dos Conceitos sobre
  Redes Neurais Convolucionais}     &      X      &   X        &         X   &            &            &            &            &            &             &             &               &      \\
  \textbf{Preparação da Base de Dados}                                     &            &      X     &      X     &            &            &            &            &            &             &             &               &      \\
  \textbf{Treinamento e teste das
  redes neurais convolucionais}          &            &            &            &      X     &     X       & X           &      X      &    X        &   X          &             &               &      \\
  \textbf{Avaliação dos Resultados}                               &            &            &            &            &            &          &            &            &             &      X       &      X         &      \\
  \bottomrule
\end{tabular}
\end{small}
\end{center}
\end{table}
