%!TEX root = ../sbc-template.tex
A realização de um trabalho de conclusão de curso desta natureza é justificada por várias razões. No contexto da interação entre telespectador e \emph{Smart} TV, um estimador de idade pode ser utilizado para facilitar a coleta de informações que contribuam para melhor experiência de provimento de conteúdo e de configurações personalizadas. Em particular, a estimação de idade dos telespectadores pode ser especialmente para a implementação de um controle parental mais eficiente, protegendo crianças e adolescentes de conteúdos inadequados à sua faixa etária.

Um outro aspecto que ressalta a importância da realização de um trabalho desta natureza é a prática e a proposição de soluções envolvendo \emph{machine learning}. Esta é uma área de vanguarda na Computação e seu potencial para resolução de problemas práticos está em franco desenvolvimento. Ao considerar a elaboração do estimador proposto, será necessário dominar conhecimentos de  ferramental tecnológico atual, o que pode colaborar na minimização da distância entre o profissional em formação e os anseios do mercado de trabalho da área.

Por fim, há que se mencionar a relação entre a área de pesquisa considerada neste trabalho de conclusão de curso e o Laboratório de Sistemas Inteligentes (LSI). Este trabalho alinha-se com os objetivos desta iniciativa do Núcleo de Computação (NUCOMP), motivando o desenvolvimento de uma solução  inovadora que utiliza técnicas da Inteligência Artificial.
