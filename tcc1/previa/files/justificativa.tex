%!TEX root = ../sbc-template.tex
A realização de um trabalho de conclusão de curso desta natureza é justificada por várias razões. No contexto da interação entre telespectador e \emph{Smart}TV, o estimador pode ser utilizado para a coleta de informações que contribuam para melhor adequação das configurações do produto e da própria programação aos seus usuários. Além disto, a tarefa de detecção automática da idade dos telespectadores pode ser utilizada para a construção um sistema de controle parental mais seguro que o que utiliza senhas, por exemplo. Acima de tudo, deve-se reservar ao usuário o direito de saber e concordar com tais políticas.

Juntamente, serão abordadas questões comuns de aprendizado de máquina, como a escolha de parâmetros e arquiteturas para modelos que se adequem a determinada tarefa de aprendizado, o \emph{tradeoff} entre \emph{bias} e variância, \emph{overfitting} e \emph{underfitting}, comparação de resultados obtidos por modelos diferentes, revisão das técnicas mais recentes presentes na literatura, entre outros.

Por fim, este trabalho envolve o desenvolvimento de uma aplicação prática envolvendo técnicas de \emph{Deep Learning}, em especial os modelos de redes neurais convolucionais profundas aplicados em visão computacional. Esta área do aprendizado de máquina é tida como o estado da arte em várias disciplinas, particularmente em visão computacional e reconhecimento de fala. Portanto, a construção de um trabalho nesta área ainda na graduação contribui para a propagação de conhecimentos contemporâneos e inovadores no ambiente acadêmico.
