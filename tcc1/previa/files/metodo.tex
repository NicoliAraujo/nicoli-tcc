%!TEX root = ../sbc-template.tex
\todo{gancho}
A metodologia para o desenvolvimento deste trabalho de conclusão de curso consiste, inicialmente, no \emph{estudo dos conceitos sobre redes neurais convolucionais}. Para tanto, considerar-se-á a literatura desta área para que haja o entendimento das bases biológicas deste modelo computacional, como funcionam, quais as características e os modelos mais importantes. Além disto, serão endereçados com mais detalhes os aspectos de treinamento destas redes, especialmente os algoritmos para convolução e \emph{max pooling} de imagens, técnicas como o ajuste fino, entre outros. Também serão estudados ambientes de desenvolvimento, bibliotecas e outras tecnologias adequadas para o modelo considerado.

<<<<<<< HEAD
Os demais passos que compõem a metodologia deste trabalho baseiam-se no \emph{fluxo de atividades de machine learning}, conforme \cite{marsland2015machine}. Inicialmente, haverá a aquisição e o pré-processamento de imagens para \emph{consolidar uma base de dados} para esta tarefa de aprendizado. Nesta etapa, será considerada a literatura e, se possível, outras bases de dados já disponíveis e com licença livre de utilização.
=======
A próxima etapa consiste na \emph{preparação da base de dados} adequada para o problema. Aqui, serão consideradas a literatura e bases de dados disponíveis na internet sob licensa livre que sejam satisfatórias para o problema proposto.
>>>>>>> parent of df6fef4... proposta + trabalhos relacionados em referencias

A seguir, será realizado o \emph{treinamento e teste das redes neurais convolucionais} para o problema em questão, considerando diferentes parâmetros de configuração. A base de dados será dividida: uma parte será utilizada para o treinamento e validação dos modelos propostos, e uma terceira parte ficará reservada para os testes de predição. A saída da rede será um número correspondente à idade de um indivíduo presente em uma imagem. Para verificar o desempenho das redes, medidas de performance serão averiguadas na etapa de teste.

Por fim, haverá a etapa de \emph{avaliação dos resultados}. Métricas como a raiz quadrada do erro médio (\emph{Root Mean Squared Error -- RMSE}) e $R^2$ serão utilizadas para mensurar a eficiência das redes à tarefa de aprendizado em questão. Se possível, serão realizadas comparações entre modelos e um ou mais modelos serão eleitos como os mais adequados ao problema.
