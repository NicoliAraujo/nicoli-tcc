%!TEX root = ../sbc-template.tex
A metodologia para o desenvolvimento deste trabalho consiste na realização da \emph{fundamentação teórica sobre machine learning}, em especial contemplando os conceitos relativos às redes neurais convolucionais. Para tanto, considerar-se-á a literatura desta área para que haja o entendimento das bases biológicas deste modelo computacional, como funcionam, quais as características e os modelos mais importantes. Neste estudo, além dos aspectos teóricos, serão considerados os ambientes de desenvolvimento, bibliotecas e outras tecnologias para implementação dos conceitos contemplados..

Os demais passos que compõem a metodologia deste trabalho baseiam-se no \emph{fluxo de atividades de machine learning}, conforme \cite{marsland2015machine}. Inicialmente, haverá a aquisição e o pré-processamento de imagens para \emph{consolidar uma base de dados} para esta tarefa de aprendizado. Nesta etapa, será considerada a literatura e, se possível, outras bases de dados já disponíveis e com licença livre de utilização.

A seguir, há a \emph{proposição de diferentes modelos de redes neurais convolucionais} para a tarefa de aprendizado em questão. Nesta etapa, serão consideradas diferentes arquiteturas, parâmetros e hiperparâmetros de configuração. Estes procedimentos visam consolidar um espaço de busca de modelos que possam endereçar a tarefa de maneira mais eficiente.

O próximo estágio consiste no \emph{treinamento das redes neurais convolucionais} para o problema em questão, considerando diferentes parâmetros de configuração. Durante este processo, uma parte da base de dados é apresentada para que haja o ajuste de pesos nas camadas que compõe o aprendizado de características necessárias para que os modelos propostos sejam capazes de realizar a tarefa de \emph{machine learning} pretendida. O treinamento das redes ocorrerá utilizando computação em núvem, tendo em vista a capacidade computacional necessária para realizar este procedimento.

Segue-se então o \emph{teste das redes}, respeitando uma abordagem de validação cruzada e utilizando métricas de desempleho apropriadas. O objetivo desta fase consiste em aferir os modelos propostos e treinados quanto à sua capacidade de generalização.

Por fim, para identificação de um modelo mais adequado à esta tarefa, as \emph{métricas de desempenho serão comparadas} e os melhores modelos elencados a partir destes valores, apontando assim um estimador apropriado para o problema inicialmente considerado.

Alem destas atividades, há que se considerar a escrita da proposta e do projeto final do trabalho de conclusão de curso, bem como as defesas parcial e final, que precisam ser incorporadas no cronograma.
