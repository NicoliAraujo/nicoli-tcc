%!TEX root = ../sbc-template.tex

A aplicação de redes neurais convolutivas em problemas de classificação e detecção de objetos em imagens têm obtido resultados significativamente positivos. Em \cite{vggnet}, \cite{resnet}, \cite{inception}, \cite{redmon2016you}, \cite{ssd} e outros, são descritas arquiteturas robustas capazes de detectar dezenas de objetos em várias situações. Treinadas com conjuntos de dados visuais que contam com milhares de exemplos como a ImageNet, Pascal VOC e COCO, estas redes são conhecidas por seu bom desempelho. Algumas destas redes foram afinadas utilizando conjuntos de dados menores e especializados para a tarefa de estimação de idade aparente.

O trabalho de \cite{rothe2015dex} relata um método para estimação de idade aparente em imagens de faces imóveis utilizando \emph{deep learning}. Propõe-se um conjunto de $20$ redes neurais convolucionais classificadoras com arquiteturas VGG-16 pré-treinadas com a base de dados visuais ImageNet, e ajustadas utilizando imagens disponibilizadas pelo IMDB, Wikipedia, e o conjunto de dados \emph{Looking At People}--LAP para anotação de idade aparente. Cada modelo tem como saída um número discreto entre $0$ e $100$, representando a idade prevista. A saída final do modelo consiste na média entre as idades previstas pelos $20$ redes. A solução atingiu um MAE (\emph{Mean Average Error}) de $3.221$ na fase de testes.

Em \cite{liu2015agenet} cria-se um estimador composto pela fusão de um modelo regressor e outro classificador. A rede neural convolucional profunda \emph{GoogLeNet} (ou \emph{Inception}) \cite{inception} sofreu modificações em sua arquitetura, como adição de normalização do batch, remoção de camadas de \emph{dropout} e perda. O conjunto de modelos conseguiu prever idades com MAE de $3.3345$.

Ademais, é possível encontrar resultados satisfatórios para a tarefa de aprendizado proposta utilizando modelos menos complexos. Com o objetivo de consolidar um método de classificação de idade e gênero, \cite{levi2015age} propõe uma rede neural convolucional de natureza mais simples, se comparada com \cite{inception}, \cite{vggnet} ou \cite{resnet}. Sua arquitetura consiste em três camadas convolucionais com \emph{dropout} e funções de ativação \emph{ReLU}, seguidas por três camadas totalmente conectadas. A camada de saída tem como função de ativação a Softmax. A escolha por um design de rede menor é motivado pelo desejo de reduzir o risco de \emph{overfitting} e pela natureza do problema, que contém apenas 8 classes de idade. O modelo é treinado utilizando apenas o conjunto de referência \emph{Adience}, composto por imagens não filtradas para classificação de idade e gênero. Considerando uma margem de erro de uma classe vizinha, a melhor rede obteve acurácia de $84.7\% \pm 2.2$ ao empregar a técnica de sobre-amostragem.
