\begin{frame}{Metodologia}
  \begin{itemize}
  	\item Utilização de Séries Temporais
    \begin{itemize}
      \item Janela de tempo: $6$ meses
      \item Precipitação, Índices Niño e TSA
      \item \emph{Time Delay Neural Networks}
      \item Atributos de entrada: 36
    \end{itemize}
  	\ \ \newline
    \item Normalização dos dados
    \begin{itemize}
      \item $\tilde N(0,1)$
      \item Amortizar os dados na mesma escala
      \item Precipitação e índices Niño
    \end{itemize}
  \end{itemize}
\end{frame}

\begin{frame}{Metodologia}
  \begin{itemize}
    \item Cálculo da \alert{anomalia}
    \begin{itemize}
      \item Diferença em relação à média histórica: $x_i - \overline{x}$
      \item Muito adotada na Meteorologia
      \item Captura noção de tendência
    \end{itemize}
  \end{itemize}
  \pause
  \begin{block}{Abordagem}
  Prever a classificação da anomalia mensal do volume de precipitações em Manaus, Amazonas
  \end{block}
\end{frame}

\begin{frame}{Metodologia}
\begin{itemize}
  \item Dimensionamento das redes neurais
  \ \ \newline
  \item Camada de entrada: $N_i = 36$
  \item Camada de saída: $N_o = 1$
  \item Camadas ocultas: 1 a 2
  \item Neurônios nas camadas ocultas: regra da pirâmide geométrica
  \begin{equation}
  N_h = \alpha \sqrt{N_i \cdot N_o},
  \end{equation}
 \item com  $0.5 \leq \alpha \leq 2$
 \ \ \newline
 \item \alert{Resultado}: $3 \leq N_h \leq 12$
 \end{itemize}
\end{frame}

\begin{frame}{Metodologia}
\begin{itemize}
\item Outros parâmetros:
\begin{itemize}
  \item Função de Ativação: Logística ou Tangente Hiperbólica
  \item Épocas: 2000
  \item Taxa de Aprendizado Inicial: $0.001$ ou $0.003$
  \item Fração de Decréscimo: $0.0001$ ou $0.01$
\end{itemize}
\end{itemize}
\begin{block}{Resultado do Dimensionamento}
  136 redes neurais a serem treinadas e testadas para cada mês do ano!
\end{block}
\end{frame}

\begin{frame}{Metodologia}
  \begin{itemize}
  \item Partição do conjunto de dados:
  \begin{itemize}
    \item De 1950 a 2000 -- 77$\%$: Treinamento
    \item 10$\%$ dos dados de treinamento reservados para validação
    \item De 2001 a 2015 -- 23$\%$: Teste
  \end{itemize}
  \ \ \newline
  \item Métrica de performance: Acurácia
  \end{itemize}
  \begin{equation}
		Acuracia(\%) = \frac{100}{n}\sum_{t=1}^n 1 - \abs{y_t - p_t}
	\end{equation}
\end{frame}
