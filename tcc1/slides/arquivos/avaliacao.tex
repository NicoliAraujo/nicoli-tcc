%!TEX root = ../main.tex
\begin{frame}{Separação dos Dados}
	\begin{itemize}
		\item Total de amostras: 65
		\item De 1950 a 2015
		\item Separação de dados:
		\begin{itemize}
			\item De 1950 a 2000 -- 77$\%$: Treinamento
			\item 10$\%$ dos dados de treinamento -- Validação
			\item De 2001 a 2015 -- 23$\%$: Teste
		\end{itemize}
		\item Dados do teste serão avaliados
		\item Cada arquitetura treinada e testada 100 vezes
	\end{itemize}
\end{frame}


\begin{frame}{Medida de Performance: Acurácia}
	\begin{itemize}
		\item Dada por:
	\end{itemize}
	\begin{equation}
		Acuracia(\%) = \frac{100}{n}\sum_{t=1}^n 1 - \abs{y_t - p_t}
	\end{equation}
	\begin{itemize}
		\item Sendo:
		\begin{itemize}
			\item $y_t$: valor real
			\item $p_t$: valor previsto
			\item n: Quantidade de exemplos
		\end{itemize}
		\item Das 100 execuções, considera-se a rede com melhor acurácia
		\item Acurácia baseia-se nos dados de teste
	\end{itemize}
\end{frame}

\begin{frame}[b]{Análise da Acurácia das Redes Neurais}
	\begin{figure}
		\includegraphics[width=0.8\textwidth]{./img/boxplot_acuracia}
		\caption{Gráfico de \emph{boxplot} para a média das acurácias obtidas.}
		 \label{fig:boxplotResultados}

	\end{figure}
\end{frame}
\begin{frame}{Análise da Acurácia das Redes Neurais}
	\begin{table}[H]
  \centering
  \begin{adjustbox}{width=1\textwidth}
  \begin{tabular}{cccccc}
    \toprule
    \textbf{Mês} &  \textbf{Arquitetura} & \textbf{Função de Ativação} &  \textbf{Taxa de Aprendizado} &   \textbf{Taxa de Decréscimo} &  \textbf{Acurácia} \\
    \midrule
    1 &     (37, 9, 1) &  Tangente Hiperbólica &  0.0001 &  0.003 &      0.93 \\
    2 &     (37, 7, 1) &             Sigmoidal &  0.0001 &  0.003 &      0.93 \\
    3 &  (37, 8, 4, 1) &  Tangente Hiperbólica &  0.0001 &  0.001 &      1.00 \\
    4 &     (37, 4, 1) &  Tangente Hiperbólica &  0.0001 &  0.001 &      0.87 \\
    5 &     (37, 8, 1) &  Tangente Hiperbólica &  0.0100 &  0.001 &      0.93 \\
    6 &     (37, 5, 1) &  Tangente Hiperbólica &  0.0100 &  0.003 &      0.87 \\
    7 &     (37, 8, 1) &  Tangente Hiperbólica &  0.0100 &  0.003 &      0.93 \\
    8 &  (37, 6, 6, 1) &             Sigmoidal &  0.0001 &  0.003 &      0.93 \\
    9 &  (37, 8, 4, 1) &  Tangente Hiperbólica &  0.0001 &  0.003 &      1.00 \\
    10 &     (37, 8, 1) &  Tangente Hiperbólica &  0.0001 &  0.003 &      0.87 \\
    11 &     (37, 8, 1) &  Tangente Hiperbólica &  0.0001 &  0.001 &      0.93 \\
    12 &  (37, 9, 3, 1) &  Tangente Hiperbólica &  0.0100 &  0.003 &      0.87 \\
    \bottomrule
  \end{tabular}
\end{adjustbox}
\caption{Redes neurais com maior acurácia para previsão do volume de precipitações em Manaus nos diferentes meses do ano.} \label{tab:melhores}
\end{table}
\end{frame}
