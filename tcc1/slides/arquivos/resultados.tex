\begin{frame}{Resultados}
 \begin{itemize}
   \item Identificação de uma rede neural para previsão do volume de precipitações em cada mês do ano
   \ \ \newline
   \item Acurácia média: $92,16\%$
   \ \ \newline
   \item Não foi possível identificar um único modelo de RNAs capaz de capturar todas as características de todos os meses
   \begin{itemize}
     \item Contraste com o trabalho de Lima \& Guedes
   \end{itemize}
 \end{itemize}
\end{frame}

\begin{frame}{Resultados}
  \begin{table}[H]
    \caption{Redes neurais com maior acurácia para previsão do volume de precipitações em Manaus nos diferentes meses do ano.} \label{tab:melhores}
    \centering
    \begin{footnotesize}
    \begin{tabular}{cccccc}
      \toprule
      \textbf{Mês} &  \textbf{Arquitetura} & \textbf{Função de Ativação} &  \textbf{Taxa de Aprendizado} &   \textbf{Taxa de Decréscimo} &  \textbf{Acurácia} \\
      \midrule
      1 &     (37, 9, 1) &  Tangente Hiperbólica &  0.0001 &  0.003 &      0.93 \\
      2 &     (37, 7, 1) &             Sigmoidal &  0.0001 &  0.003 &      0.93 \\
      3 &  (37, 8, 4, 1) &  Tangente Hiperbólica &  0.0001 &  0.001 &      1.00 \\
      4 &     (37, 4, 1) &  Tangente Hiperbólica &  0.0001 &  0.001 &      0.87 \\
      5 &     (37, 8, 1) &  Tangente Hiperbólica &  0.0100 &  0.001 &      0.93 \\
      6 &     (37, 5, 1) &  Tangente Hiperbólica &  0.0100 &  0.003 &      0.87 \\
      7 &     (37, 8, 1) &  Tangente Hiperbólica &  0.0100 &  0.003 &      0.93 \\
      8 &  (37, 6, 6, 1) &             Sigmoidal &  0.0001 &  0.003 &      0.93 \\
      9 &  (37, 8, 4, 1) &  Tangente Hiperbólica &  0.0001 &  0.003 &      1.00 \\
      10 &     (37, 8, 1) &  Tangente Hiperbólica &  0.0001 &  0.003 &      0.87 \\
      11 &     (37, 8, 1) &  Tangente Hiperbólica &  0.0001 &  0.001 &      0.93 \\
      12 &  (37, 9, 3, 1) &  Tangente Hiperbólica &  0.0100 &  0.003 &      0.87 \\
      \bottomrule
    \end{tabular}
  \end{footnotesize}
  \end{table}
\end{frame}
