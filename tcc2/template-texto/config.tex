%!TEX root = novoIndex.tex
\usepackage[utf8]{inputenc}
\usepackage[T1]{fontenc}
\usepackage{ae}
\usepackage[brazil]{babel}
\usepackage{a4wide}
\usepackage{comment}
\usepackage{stackengine}
\usepackage{adjustbox}
%\usepackage{subfig}
\usepackage{caption, subcaption}
\usepackage[pdftex]{color}
\usepackage{longtable}
\usepackage{float}
\usepackage{rotating}
\usepackage{varwidth}
\usepackage{fancyvrb}
\usepackage{fancyhdr}
\usepackage{setspace}
\usepackage{lscape}
\usepackage{textcase}
\usepackage{anysize}
\usepackage{booktabs}
\usepackage{cite}
\usepackage{bbm}
\usepackage{amsmath}
\usepackage{amssymb}
\usepackage{amsfonts}
\usepackage{amsthm}
\usepackage{dsfont}
%% Cores, fontes e afins
\usepackage[colorlinks,linkcolor=black,urlcolor=black,citecolor=black]{hyperref}

\usepackage{xcolor}
\definecolor{lightblue}{RGB}{0,191,255}
\usepackage[textsize=tiny,backgroundcolor=lightblue,linecolor=lightblue]{todonotes}

\usepackage[portuguese,ruled,lined]{algorithm2e}
\usepackage{algorithmic}
\usepackage{scalefnt}

\usepackage[alf]{abntex2cite}
\marginsize{20mm}{20mm}{20mm}{15mm}


%% Cabe�alhos
\renewcommand{\topfraction}{1}
\renewcommand{\bottomfraction}{1}
\renewcommand{\floatpagefraction}{1}
\renewcommand{\textfraction}{0}
%\renewcommand{\baselinestretch}{2}
\doublespacing %espa�amento duplo
\sloppy

%% Nomes
\floatstyle{plain}  %%% tipos: plain, boxed, ruled
\newfloat{codigo}{tbp}{lop}[section]
\floatname{codigo}{C�digo}

%%% nome para ser usado no sum�rio

\newcommand{\listofcodename}{Lista de C\'{o}digos}



% RESUMO ----------------------------------------------------------------------------------------------------------------------------------------------------------------------

\newcommand{\resumo}[1]{
\begin{center} \LARGE \bf Resumo \end{center}

\vskip 4em
\input{#1}

\newpage

}

% ABSTRACT ----------------------------------------------------------------------------------------------------------------------------------------------------------------------

\newcommand{\abstractt}[1]{
\begin{center} \LARGE \bf Abstract \end{center}

\vskip 4em
\input{#1}

\newpage

}

% Sum�rio -----------
\newcommand{\sumario}{
\renewcommand{\contentsname}{Sum\'{a}rio}
\tableofcontents
\addcontentsline{toc}{chapter}{\listtablename}
\listoftables

\newpage
\addcontentsline{toc}{chapter}{\listfigurename}
\listoffigures
\addcontentsline{toc}{chapter}{\listofcodename}
\listof{codigo}{\listofcodename}  % Lista de C�digos

\clearpage
}
