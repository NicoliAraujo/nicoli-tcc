%!TEX root = ../../novoIndex.tex

Para a realização deste trabalho são utilizadas tecnologias comumente relacionadas às praticas de ML. A linguagem de programação adotada é o Python 3, em conjunto com bibliotecas como a \emph{Python Data Analysis Library} (Pandas) \cite{pandas} utilizada para análise e consolidação de conjuntos de dados, as bibliotecas de visualização de dados Seaborn \cite{seaborn} e MatPlotLib \cite{matplotlib}, as bibliotecas de ML com suporte a DL Keras \cite{keras} e TensorFlow \cite{tensorflow}.

A fim de executar o treinamento e teste dos modelos propostos, utilizou-se uma instância de máquina virtual disponibilizada através da \emph{Google Compute Engine} (GCE) \cite{gce}, dotada de $4$ núcleos de processamento com $4$ GHz cada e $16$ GB de memória RAM. A GCE é parte da \emph{Google Cloud Platform}, uma suíte de computação em nuvem oferecida pelo \emph{Google}. Utilizou-se também um desktop físico pertencente à infraestrutura do LSI -- UEA, com processador Intel Core I7 8a geração, 16 GB de RAM, 480 GB de memória SSD, 2 TB de memória HD e duas placas de fídeo NVIDIA GeForce 1080 com 11 GB de memória cada.
