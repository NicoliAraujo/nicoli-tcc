%!TEX root = ../../novoIndex.tex
Este trabalho teve por objetivo elaborar estratégias inteligentes para estimação de idade de telespectadores de  \emph{Smart} TVs a partir de suas respectivas fotografias faciais. Para endereçar este problema, consolidou-se um base de dados contendo $47.950$ exemplos de faces centralizadas e com as respectivas idades dos sujeitos retratados. Estes dados foram então utilizados para treinar e testar redes neurais convolucionais para a tarefa de Aprendizado de Máquina considerada segundo uma abordagem de validação cruzada do tipo \emph{Holdout}.

Considerando o modelo adotado, foram conduzidas várias abordagens de treino e teste utilizando arquiteturas canônicas de CNNs, a citar: LeNet, AlexNet, VGG-16 e SqueezeNet, esta última, em particular, oriunda de um trabalho recente na literatura. Houve uma adaptação na camada de saída destas redes para uma tarefa de regressão e variou-se o parâmetro de função de ativação para os valores \emph{ReLU} e \emph{Leaky ReLU}. Considerou-se também técnicas de normalização da entrada, equalização de histograma e ainda \emph{data augmentation}. Estas redes, parâmetros, hiperparâmetros e técnicas foram explorados de maneira sistemática em nove cenários distintos.

De maneira geral, não foi possível constatar uma progressiva melhora ao passo que as abordagens foram conduzidas, fato decorrente dos desafios na busca de soluções ideias em redes neurais quando considerado o aumento no espaço de busca. Apesar disso, o trabalho foi conduzido conforme práticas de natureza heurística sugeridas por diversos autores da literatura, considerando modelos compatíveis com o estado da arte. Os resultados obtidos mostraram que a melhor CNN para esta tarefa foi X sujeita à abordagem Y, a qual considerou a entrada de dados sem/com \emph{data augmentation} e Z e W. \todo{Completar aqui conforme os resultados}.

Embora os resultados aqui apresentados não tenham superado o estado da arte na literatura para este problema, pretensão não esperada para o escopo de um trabalho de conclusão de curso em face à comparação com trabalhos de níveis mais avançados, os resultados obtidos ressaltam a condução assertiva da metodologia, ilustram a prática de métodos e tecnologias desta área de pesquisa emergente e propõem um patamar para comparação com outros autores que não utilizam técnicas de extração de características nem tampouco \emph{ensembles} de CNNs, em contraste como os melhores resultados observados atualmente.

Como trabalhos futuros considera-se a abordagem deste problema como uma tarefa de classificação e a avaliação do impacto da utilização de técnicas de \emph{transfer learning}. Há ainda a possibilidade de uma análise mais detalhada dos grupos de idade com maior e menor índice de erro, na tentativa de obter estimadores mais apropriados considerando as características de cada grupo.  Continuar endereçando este problema é de suma importância do ponto de vista prático para o desenvolvimento de diversas soluções de recomendação de conteúdo e controle parental em \emph{Smart} TVs, auxiliando no desenvolvimento destas soluções tecnológicas. 
