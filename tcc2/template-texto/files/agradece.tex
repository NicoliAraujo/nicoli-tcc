Agradeço primeiramente à minha mãe, Francisca Dones Nicácio Pinheiro, por ter me incentivado durante toda a vida a buscar uma formação que traga realização profissional. Agradeço pelo apoio incondicional dado à minha educação, da maneira que apenas pais dedicados conseguem fazer. Reconheço também o esforço pela conscientização e motivação pela busca por uma realidade melhor, tanto financiera quanto intelectualmente. Apesar de eu não ser uma pessoa religiosa, agradeço a minha mãe por orar por mim, gesto de carinho sempre bem vindo.

Agradeço também aos outros membros da minha família, em especial minha irmã Vitória Gabriela Pinheiro de Araujo e meu pai Valdenor Oliveria de Araujo, ambos fundamentais para a estabilidade psicológica e financeira necessária para finalizar uma graduação com turno integral.

Agradeço à Profa. Dra. Elloá B. Guedes, que em 2014 aceitou me direcionar durante a monitoria de Algoritmos e Programação I, e tem me orientado desde então. Grande mentora, agradeço por todas as liçoes ensinadas, desde as mais complexas, ministradas durante as aulas da antiga Fundamentos de Engenharia de Computação II até as mais triviais. Peço perdão por não ser capaz de aprender todas, e tempo para aprender mais algumas.

Agradeço aos meus colegas de curso, incontáveis, que me ajudaram em diversos momentos, sob diversas circunstâncias, a persitir em busca da graduação. Em especial, agradeço a Giovana Oliveira de Lucca e Janderson do Nascimento Lira, fundamentais defensores dos interesses dos alunos durante a troca das grades curriculares do curso de Engenharia de Computação. Agradecimentos especiais ao DNCG, a caminhada teria sido bem mais dura sem vocês, obrigada por estarem lá em todos os momentos, pelo apoio, por me ajudarem a ser um ser humano e uma profissional melhor.

Agradeço ao Núcleo de Computação e a todos os professores e coordenadores que proporcionaram imenso aprendizado durante a graduação. Agradeço também a Universidade do Estado do Amazonas, seus servidores, prestadores de serviço, alunos, e outros envolvidos no funcionamento desta instituição de ensino, por toda a infraestrutura fornecida, desde os processos de seleção até o fim da graduação. Agradeço também ao Governo do Estado do Amazonas manter a universidade em condição útil à sociedade durante o tempo da minha formação.

Agradeço ao setor de Pesquisa e Desenvolvimento da Envision Ind. de Prod, Eletrônicos LTDA. por oferecerem um ambiente de convívio profissional em que os conceitos aprendidos em sala de aula e laboratório são efetivamente usados para endereçar problemas práticos e desenvolver soluções para produtos. Obrigada por dar o apoio profissioal necessário para a minha graduação. Agradeço especialmente a Ruan Belém por incentivar a abordagem do tema central deste trabalho e por demonstrar suas aplicações práticas e relevância.

Agradeço à Fundação de Amparo à Pesquisa do Estado do Amazonas (FAPEAM) que, por meio do Projeto PROTI Pesquisa 11/2017, colaborou para a consolidação da infraestrutura física e tecnológica do Laboratório de Sistemas Inteligentes da Escola Superior de Tecnologia da Universidade do Estado do Amazonas. Este trabalho de conclusão de curso é um dos produtos deste projeto, pois foi desenvolvido no referido laboratório, fez uso dos recursos  computacionais ali disponíveis e foi melhorado graças às discussões e interações com o grupo de pesquisa nele sediado.
