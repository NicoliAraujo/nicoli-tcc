%!TEX root = ../novoIndex.tex

A face humana detém diversas informações relativas à caracterização de indivíduos. Alguns atributos faciais importantes cuja percepção pode revelar aspectos relacionadas a tratos individuais são identidade, expressão facial, gênero, idade, origem étnica e pose. Neste contexto, seres humanos são capazes de determinar a idade de indivíduos entre 20 e 60 anos a partir da face com alta acurácia.

Porém, uma máquina pode se sair tão bem quanto um humano nest tarefa?
A estimação de idade consiste de rotular uma imagem automáticamente com a idade exata ou um intervalo de idade que corresponda ao grupo de idade de uma face de um indivíduo.
O aumento de aplicações para a estimação da idade tem estimulado o interesse na área.
A estimação automática de idade através de fotografias facias tem diversas aplicações.
A idade real consiste nos anos acumulados desde o nascimento de um indivíduo. Já a idade aparente consiste na informação de idade mostrada através da aparência visual. A idade estimada é a idade de um indivíduo reconhecida por uma máquina através da aparência visual. Por fim, a idade percebida é a idade de um indivíduo aferida por seres humanos a partir da aparência visual.

Apesar de ser típicamente consistente com a idade atual, a idade aparente pode não corresponder com a real. Estas variações inevitáveis são típicamente causadas por diferenças genéticas  entre diferentes indivíduos e fatores artificiais e de ambiente. Ambas as idades percebidas e estimadas são definidas a partir da idade aparente. A idade real é comumente definida como a verdadeira.

A estimação de idade realizada automaticamente é útil em aplicações onde não é necessário identificar específicamente o indivíduo, mas é necessário saber a idade.

Problemas
Considerando que diferentes grupos de clientes têm diferentes hábitos de consumo, preferências, responsividade, e expectativas de compras, companhias podem ganhar mais lucos ao saber os grupos de clientes que tem, respondendo diretamente a todas as necessidades específicas dos clientes e provendo produtos ou serviços customizados. A parte mais desafiadora é obter e analizar uma quantidade suficiente de informações pessoais de todos os perfis de clientes, o que faz necessário que companhias estabeleçam longas relações com consumidores e garantam uma alta taja de custo de entrada.

No caso de \emph{Smart} TVs,
é essencial que estes aparelhos sejam capazes de capturar o perfil e o interesse dos seus telespectadores a fim de oferecer uma experiência mais rica.
A recomendação de conteúdo, por exemplo, pode levar em conta características individuais, tais como idade e gênero. Porém, se fornecidos de maneira habitual, via preenchimento de formulários, além de ser uma tarefa massante, pode não refletir de maneira realística o perfil individual dos vários usuários que podem estar à frente de uma \emph{Smart} TV em um determinado momento.
%
\emph{Smart} TVs possuem câmeras que podem ser habilitadas para aquisição de imagens daqueles que estão à frente do televisor, respeitadas as preferências de privacidade de cada usuário. É possível usá-las como entrada para sistemas inteligentes de identificação de características, cujas previsões podem ser usadas, por exemplo, para recomendação de conteúdo. No caso da idade, em particular, é possível usar estas informações para realizar um controle parental mais eficiente, protegendo crianças e adolescentes de conteúdos inadequados à sua faixa etária.\cite{Guardian:CameraSmartv}.

Outros casos em que isto é aplicado inclui
um dono de loja de fast food pode querer saber qual a porcentagem de cada grupo de idade prefere e compra que tipo de sanduíche. Anunciantes querem atingir audiências específicas de clientes potenciais, para anúnicios específicos em termos de grupos de idade. Uma companhia de smartphones pode querer saber qual grupo de idade é mais interessado nos seus novos modelos de produtos mostrados em um quiosque público. Um painel (display) de uma loja pode mostrar um terno enquanto um adulto passa ou uma calça jeans enquanto um adolescente passa. Obviamente é quase impossível atingir estes resultados por conta de problemas de privacidade. No entanto, com a ajuda de um sistema de estimação automática de idade, uma câmera tirando fotos de consumidores poderia coletar dados demográficos ao capturar imagens da face dos clientes e automaticamente rotular grupos de idade. Tudo isto pode ser feito sem violar a privacidade de ninguém.

Casos de controle acesso e monitoramento de vigilância são mais cruciais em na nossa vida diária, especialment quando tecnologias avançadas e informações exposlivas se tornam comuns de se acessr e possuir. Com uma câmera como entrada, um sistema de estimação de idade pode alertar ou parar menores de idade de entrarem em bares ou lojas de vinho, previnir menores de comprar produtos com tabaco de máquinas de venda, recusar idosos quando eles quiserem tentar ir de montanha russa em parques de diversão, e para crianças de acessar conteúdo adulto na internet ou em filmes restritos. No Japão, policiais descobriram que um grupo de idade específico é mais apto a realizar fraudes de transferência de dinheiro em caixas eletrônicos, em que estimação de idade em monitoramento de vigilância pode ter um papel importante. Softwares de estimação de idade também podem ser utilizados em sistemas de saúde, como enfermeiras robóticas e unidades de tratamento intensivo inteligentes. para serviços customizados. Por exemplo, um avatar personalizado será selecionado automáticamente do conjunto de avatares customizados para interagir com pacientes de diferentes grupos de idade com preferências partibularesl.

Diante do que foi exposto, este trabalho de conclusão de curso considera o desenvolvimento de estratégias inteligentes, baseadas na utilização de técnicas de \emph{Deep Learning}, para estimação da idade de telespectadores a partir de fotografias faciais. Embora a estimação de outras características também pudesse ser realizada mediante a análise de fotografias faciais, desde gênero até a presença de doenças, optou-se pela idade por ser um atributo comum a todos os indivíduos, pelo potencial de aplicações, pela existência de bases de dados adequadamente rotuladas com este atributo e pelo menor potencial de infringência das searas privadas dos usuários.

\section{Objetivos}\label{sec:objetivo}
%!TEX root = ../sbc-template.tex
O objetivo geral deste trabalho consiste em propor um estimador de idade para telespectadores de \emph{Smart} TVs. Para alcançar esta meta, alguns objetivos específicos precisam ser contemplados, a citar:

\begin{itemize}
     \item Formular um referencial teórico sobre redes neurais convolucionais, modelo de \emph{machine learning} considerado, contemplando suas características, principais arquiteturas, métodos de treinamento e teste;
     \item Consolidar uma base de dados para a tarefa de \emph{machine learning} proposta, contemplando exemplos realísticos;
     \item Identificar tecnologias adequadas para implementar o estimador proposto;
     \item Propor, treinar e testar diferentes arquitteturas  de redes neurais convolucionais para a tarefa em questão;
     \item Avaliar comparativamente os estimadores propostos.
\end{itemize}


\section{Justificativa}\label{sec:just}
%!TEX root = ../../novoIndex.tex
A realização de um trabalho de conclusão de curso desta natureza é justificada por várias razões. No contexto da interação entre telespectador e \emph{Smart} TV, um estimador de idade pode ser utilizado para facilitar a coleta de informações que contribuam para melhor experiência de provimento de conteúdo e de configurações personalizadas. Em particular, a estimação de idade dos telespectadores pode ser especialmente empregada na implementação de um controle parental mais eficiente, protegendo crianças e adolescentes de conteúdos inadequados à sua faixa etária.

Um outro aspecto que ressalta a importância da realização de um trabalho desta natureza é a prática e a proposição de soluções envolvendo \emph{Machine Learning}. Esta é uma área de vanguarda na Computação e seu potencial para resolução de problemas práticos está em franco desenvolvimento. Ao considerar a elaboração do estimador proposto, será necessário dominar conhecimentos de ferramental tecnológico atual, o que pode colaborar na minimização da distância entre o profissional em formação e os anseios do mercado de trabalho da área.

Por fim, há que se mencionar a relação entre a área de pesquisa considerada neste trabalho de conclusão de curso e o Laboratório de Sistemas Inteligentes (LSI). Este trabalho alinha-se com os objetivos desta iniciativa do Núcleo de Computação (NUCOMP), motivando o desenvolvimento de uma solução inovadora que utiliza técnicas da Inteligência Artificial.


\section{Metodologia}\label{sec:metodo}
%!TEX root = ../../novoIndex.tex
A metodologia para o desenvolvimento deste trabalho consistiu na realização da \emph{fundamentação teórica sobre Machine Learning}, em especial contemplando os conceitos relativos às redes neurais convolucionais. Para tanto, considerou-se a literatura desta área para que haja o entendimento das bases matemáticas deste modelo computacional, como funcionam, quais as características e as arquiteturras mais importantes. Neste estudo, além dos aspectos teóricos, foram considerados os ambientes de desenvolvimento, bibliotecas e outras tecnologias para implementação dos conceitos contemplados.

Os demais passos que compõem a metodologia deste trabalho baseiam-se no \emph{fluxo de atividades de machine learning} \cite{marsland2015machine}. Inicialmente, houve a aquisição e o pré-processamento de imagens para \emph{consolidar uma base de dados} para esta tarefa de aprendizado. Nesta etapa, foi considerada a literatura e uma base de dados já disponível e apropriadamente anotada, com licença livre de utilização.

A seguir, houve a \emph{proposição de diferentes modelos de redes neurais convolucionais} para a tarefa de aprendizado considerada. Nesta etapa, foram elencados diferentes parâmetros e hiperparâmetros de configuração, bem como arquiteturas. Estes procedimentos visaram consolidar um espaço de busca de modelos que possam endereçar a tarefa de maneira mais eficiente.

O próximo estágio consistiu no \emph{treinamento das redes neurais convolucionais} para o problema em questão. Durante este processo, uma parte da base de dados foi apresentada aos modelos para que houvesse o ajuste de pesos, compreendendo o aprendizado das características relevantes. O treinamento das redes ocorreu utilizando computação em nuvem e computadores disponíveis no Laboratório de Sistemas Inteligentes (LSI), tendo em vista a infra-estrutura de hardware necessária para realizar este procedimento.

Seguiu-se então o \emph{teste das redes}, respeitando uma abordagem de validação cruzada e utilizando métricas de desempenho apropriadas. O objetivo desta fase consistiu em aferir os modelos propostos e treinados quanto à sua capacidade de generalização.

Por fim, para identificação de um modelo mais adequado à esta tarefa, as \emph{métricas de desempenho foram comparadas} e os melhores modelos elencados a partir destes valores, apontando assim um estimador apropriado para o problema inicialmente considerado.

Alem destas atividades, há que se considerar a escrita da proposta e do projeto final do trabalho de conclusão de curso, bem como as defesas parcial e final.


\section{Cronograma}\label{sec:crono}
%!TEX root = ../sbc-template.tex

O cronograma de realização das atividades pode ser visto na Tabela \ref{tab:cronograma}. As atividades listadas possuem relação com a metodologia detalhada na seção anterior, compreendendo os requisitos elementares para a realização deste trabalho.
\newline

\begin{table}{H}
\scalefont{0.8}
\caption{Cronograma de atividades levando em consideração os dez meses (de $02/2018$ a $12/2018$) para a realização do TCC.}
\label{tab:cronograma}

\begin{center}
\begin{small}
\begin{tabular}{p{5cm}cccccccccccc}
  \toprule
  & &  &  & &  & \textbf{2018}  & &  &  &  &  & \\
                                        & \textbf{02} & \textbf{03} & \textbf{04} & \textbf{05} & \textbf{06} & \textbf{07} & \textbf{08} & \textbf{09} & \textbf{10} & \textbf{11} & \textbf{12} \\
  \midrule
  \textbf{Escrita da Proposta}          &      X      &      X      &      X      &      X      &      X      &             &             &             &             &             &             \\
  \textbf{Fundamentação Teórica sobre
  Machine Learning}                     &      X      &      X      &      X      &      X      &             &             &             &             &             &             &             \\
  \textbf{Consolidação da Base de Dados}&             &      X      &      X      &             &             &             &             &             &             &             &             \\
  \textbf{Proposição de Modelos de
  Redes Neurais Convolucionais}         &             &             &             &      X      &      X      &      X      &      X      &      X      &             &             &             \\
  \textbf{Defesa da Proposta}          &             &             &             &             &      X      &             &             &             &             &             &             \\
  \textbf{Escrita do Trabalho Final}    &             &             &             &             &             &      X      &      X      &      X      &      X      &      X      &      X      \\
  \textbf{Treinamento das
  Redes Neurais Convolucionais}         &             &             &             &             &      X      &      X      &      X      &      X      &      X      &      X       &            \\
  \textbf{Teste das Redes
  Neurais Convolucionais}               &             &             &             &             &      X      &      X      &      X      &      X      &      X      &      X       &     X      \\
  \textbf{Comparação de Metricas
  de Desempenho}                        &             &             &             &             &             &      X      &      X      &      X      &      X      &      X      &      X      \\
  \textbf{Defesa do Trabalho Final}     &             &             &             &             &             &             &             &             &             &             &      X      \\
  \bottomrule
\end{tabular}
\end{small}
\end{center}
\end{table}


\section{Organização do Documento}
Para a apresentação desta proposta de trabalho de conclusão de curso, o presente documento está organizado como segue. Inicialmente, uma fundamendação teórica pode ser vista na Seção \ref{sec:fund_teorica}. Uma análise dos trabalhos relacionados encontra-se na Seção \ref{sec:trab_relac}. Na Seção \ref{sec:solucao_proposta} detalha-se uma solução proposta para a tarefa endereçada. Na Seção \ref{sec:resultados} estão os resultados obtidos. Finalmente, as considerações finais podem ser encontradas na Seção \ref{sec:consid_finais}.
