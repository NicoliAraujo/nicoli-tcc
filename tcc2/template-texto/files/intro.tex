%!TEX root = ../novoIndex.tex

As \emph{Smart} TVs, dispositivos resultantes da evolução tecnológica dos aparelhos de televisão domésticos, destacam-se por sua capacidade de conexão à internet e de transmissão de conteúdos advindos de outros dispositivos eletrônicos \cite{samsung:smarttv,perakakis2015proposed}.
Segundo a Pesquisa Nacional por Amostra de Domicílios (PNAD) realizada pelo IBGE em 2015, existem $16$ milhões de \emph{Smart} TVs em residências e pontos comerciais no Brasil, cujos $94\%$ foram adquiridos entre $2014$ e $2015$. Estes aparelhos foram responsáveis por $68,2\%$ do total de televisores vendidos no primeiro semestre de $2017$ \cite{pnad2015}.
%}

%As \emph{Smart} TVs são o resultado da evolução tecnológica junto aos aparelhos de televisão domésticos. Possuem capacidades interativas ligadas à internet, acesso a conteúdo online, \emph{e-commerce} de conteúdo televisivo, navegação web e acesso a redes sociais. Estes aparelhos podem ser equipados com câmeras e microfones e são capazes de transmitir conteúdo 2D e 3D \cite{samsung:smarttv,perakakis2015proposed}.

%Segundo a Pesquisa Nacional por Amostra de Domicílios (PNAD) realizada pelo IBGE em 2015, foi observado um total de $103$ milhões de aparelhos de televisões em residências e pontos comerciais, das quais $16$ milhões são de \emph{Smart} TVs. A pesquisa detalha que $94\%$ destas \emph{Smart} TVs foram adquiridas entre $2014$ e $2015$. Os números mostram um posterior aumento nas vendas de aparelhos televisores deste tipo, representando $68,2\%$ do total de televisores vendidos no primeiro semestre de $2017$ \cite{pnad2015}.

Este aumento de vendas tem várias causas, das quais destacam-se os muitos benefícios resultantes do uso de \emph{Smart} TVs quando comparadas aos aparelhos convencionais \cite{shin2013smart,differencebetween}. Em especial, cita-se o aumento da qualidade na transmissão, utilização de aplicativos diversos, a capacidade de conexão com dispositivos a exemplo de \emph{smartphones} e \emph{players} de mídia digital, a possibilidade de acesso a conteúdo \emph{online} e \emph{on demand}, gratuitos ou mediante assinaturas. Além destes benefícios, cuja maioria é resultante da conectividade com a internet, outros fatores têm justificado o aumento das vendas e do interesse do público consumidor pelas \emph{Smart} TVs, tais como o encerramento da transmissão de sinal analógico da televisão aberta, a Copa do Mundo 2018 e a tecnologia 4K \cite{leiajabuscasmart,correiopnad,estadao:explosaovideosonline}.

Considerando a grande difusão das \emph{Smart} TVs nos lares brasileiros, é essencial que estes aparelhos sejam capazes de capturar o perfil e o interesse dos seus telespectadores a fim de oferecer uma experiência mais rica. A recomendação de conteúdo, por exemplo, pode levar em conta características individuais, tais como idade e gênero. Porém, se fornecidos de maneira habitual, via preenchimento de formulários, além de ser uma tarefa massante, pode não refletir de maneira realística o perfil individual dos vários usuários que podem estar à frente de uma \emph{Smart} TV em um determinado momento.

Apesar das dificuldades práticas mencionadas, é interessante notar que muitas \emph{Smart} TVs possuem dispositivos para captura de imagens, como câmeras, pois também costumam dispor de aplicações para troca de mensagens de vídeo \cite{Guardian:CameraSmartv}. Respeitadas as preferências de privacidade de cada usuário, se estas câmeras forem habilitadas para aquisição de imagens daqueles que estão à frente do televisor, então é possível usá-las como entrada para sistemas inteligentes de identificação de características, cujas previsões podem ser usadas, por exemplo, para recomendação de conteúdo. No caso da idade, em particular, é possível usar estas informações para realizar um controle parental mais eficiente, protegendo crianças e adolescentes de conteúdos inadequados à sua faixa etária.

Diante do que foi exposto, esta proposta de trabalho de conclusão de curso considera o desenvolvimento de estratégias inteligentes, baseadas na utilização de técnicas de \emph{Deep Learning}, para estimação da idade de telespectadores a partir de fotografias faciais. Embora a estimação de outras características também pudesse ser realizada mediante a análise de fotografias faciais, desde gênero até a presença de doenças, optou-se pela idade por ser um atributo comum a todos os telespectadores, pelo potencial de aplicações, pela existência de bases de dados adequadamente rotuladas com este atributo e pelo menor potencial de infringência das searas privadas dos usuários.

\section{Objetivos}\label{sec:objetivo}
O objetivo geral deste trabalho consiste de apresentar um estimador de idade de telespectadores para aplicações de sugestão de conteúdo em \emph{Smart} TVs que utilize redes neurais convolucionais. Para alcançar esta meta, alguns objetivos específicos precisam ser contemplados, a citar:

\begin{itemize}
     \item Formular um referencial teórico sobre as redes neurais convolucionais, incluindo características, principais modelos, métodos de treinamento, validação e testes considerados para esta tarefa.
     \item Identificar, adequar e descrever bases de dados que possam ser utilizadas para implementar um estimador de idade através de imagens.
     \item Selecionar modelos e tecnologias adequados para o problema.
     \item Treinar, validar e testar diferentes redes neurais convolucionais com as bases de dados relevantes para o domínio do problema.
     \item Analisar e avaliar as redes neurais convolucionais construídas, estabelecendo métricas qualitativas e quantitativas.
\end{itemize}


\section{Justificativa}\label{sec:just}
%!TEX root = ../sbc-template.tex
A realização de um trabalho de conclusão de curso desta natureza é justificada por várias razões. No contexto da interação entre telespectador e \emph{Smart}TV, o estimador pode ser utilizado para a coleta de informações que contribuam para melhor adequação das configurações do produto e da própria programação aos seus usuários. Além disto, a tarefa de detecção automática da idade dos telespectadores pode ser utilizada para a construção um sistema de controle parental mais seguro que o que utiliza senhas, por exemplo. Acima de tudo, deve-se reservar ao usuário o direito de saber e concordar com tais políticas.

Juntamente, serão abordadas questões comuns de aprendizado de máquina, como a escolha de parâmetros e arquiteturas para modelos que se adequem a determinada tarefa de aprendizado, o \emph{tradeoff} entre \emph{bias} e variância, \emph{overfitting} e \emph{underfitting}, comparação de resultados obtidos por modelos diferentes, revisão das técnicas mais recentes presentes na literatura, entre outros.

Por fim, este trabalho envolve o desenvolvimento de uma aplicação prática envolvendo técnicas de \emph{Deep Learning}, em especial os modelos de redes neurais convolucionais profundas aplicados em visão computacional. Esta área do aprendizado de máquina é tida como o estado da arte em várias disciplinas, particularmente em visão computacional e reconhecimento de fala. Portanto, a construção de um trabalho nesta área ainda na graduação contribui para a propagação de conhecimentos contemporâneos e inovadores no ambiente acadêmico.


\section{Metodologia}\label{sec:metodo}
%!TEX root = ../sbc-template.tex
\todo{gancho}
A metodologia para o desenvolvimento deste trabalho de conclusão de curso consiste, inicialmente, no \emph{estudo dos conceitos sobre redes neurais convolucionais}. Para tanto, considerar-se-á a literatura desta área para que haja o entendimento das bases biológicas deste modelo computacional, como funcionam, quais as características e os modelos mais importantes. Além disto, serão endereçados com mais detalhes os aspectos de treinamento destas redes, especialmente os algoritmos para convolução e \emph{max pooling} de imagens, técnicas como o ajuste fino, entre outros. Também serão estudados ambientes de desenvolvimento, bibliotecas e outras tecnologias adequadas para o modelo considerado.

<<<<<<< HEAD
Os demais passos que compõem a metodologia deste trabalho baseiam-se no \emph{fluxo de atividades de machine learning}, conforme \cite{marsland2015machine}. Inicialmente, haverá a aquisição e o pré-processamento de imagens para \emph{consolidar uma base de dados} para esta tarefa de aprendizado. Nesta etapa, será considerada a literatura e, se possível, outras bases de dados já disponíveis e com licença livre de utilização.
=======
A próxima etapa consiste na \emph{preparação da base de dados} adequada para o problema. Aqui, serão consideradas a literatura e bases de dados disponíveis na internet sob licensa livre que sejam satisfatórias para o problema proposto.
>>>>>>> parent of df6fef4... proposta + trabalhos relacionados em referencias

A seguir, será realizado o \emph{treinamento e teste das redes neurais convolucionais} para o problema em questão, considerando diferentes parâmetros de configuração. A base de dados será dividida: uma parte será utilizada para o treinamento e validação dos modelos propostos, e uma terceira parte ficará reservada para os testes de predição. A saída da rede será um número correspondente à idade de um indivíduo presente em uma imagem. Para verificar o desempenho das redes, medidas de performance serão averiguadas na etapa de teste.

Por fim, haverá a etapa de \emph{avaliação dos resultados}. Métricas como a raiz quadrada do erro médio (\emph{Root Mean Squared Error -- RMSE}) e $R^2$ serão utilizadas para mensurar a eficiência das redes à tarefa de aprendizado em questão. Se possível, serão realizadas comparações entre modelos e um ou mais modelos serão eleitos como os mais adequados ao problema.


\section{Cronograma}\label{sec:crono}
%!TEX root = ../sbc-template.tex

O cronograma de realização das atividades pode ser visto na Tabela \ref{tab:cronograma}. As atividades listadas possuem relação com a metodologia detalhada na seção anterior, compreendendo os requisitos elementares para a realização deste trabalho.
\newline

\begin{table}
\scalefont{0.8}
\caption{Cronograma de atividades levando em consideração os dez meses (de $02/2018$ a $12/2018$) para a realização do TCC.}
\label{tab:cronograma}

\begin{center}
\begin{small}
\begin{tabular}{p{5cm}cccccccccccc}
  \toprule
  & &  &  & &  & \textbf{2018}  & &  &  &  &  & \\
                                        & \textbf{02} & \textbf{03} & \textbf{04} & \textbf{05} & \textbf{06} & \textbf{07} & \textbf{08} & \textbf{09} & \textbf{10} & \textbf{11} & \textbf{12} \\
  \midrule
  \textbf{Escrita da Proposta}          &      X      &      X      &      X      &      X      &      X      &             &             &             &             &             &             \\
  \textbf{Fundamentação Teórica sobre
  Machine Learning}                     &      X      &      X      &      X      &      X      &             &             &             &             &             &             &             \\
  \textbf{Consolidação da Base de Dados}&             &      X      &      X      &             &             &             &             &             &             &             &             \\
  \textbf{Proposição de Modelos de
  Redes Neurais Convolucionais}         &             &             &             &      X      &      X      &      X      &      X      &      X      &             &             &             \\
  \textbf{Defesa da Proposta}          &             &             &             &             &      X      &             &             &             &             &             &             \\
  \textbf{Escrita do Trabalho Final}    &             &             &             &             &             &      X      &      X      &      X      &      X      &      X      &      X      \\
  \textbf{Treinamento das
  Redes Neurais Convolucionais}         &             &             &             &             &      X      &      X      &      X      &      X      &      X      &      X       &            \\
  \textbf{Teste das Redes
  Neurais Convolucionais}               &             &             &             &             &      X      &      X      &      X      &      X      &      X      &      X       &     X      \\
  \textbf{Comparação de Metricas
  de Desempenho}                        &             &             &             &             &             &      X      &      X      &      X      &      X      &      X      &      X      \\
  \textbf{Defesa do Trabalho Final}     &             &             &             &             &             &             &             &             &             &             &      X      \\
  \bottomrule
\end{tabular}
\end{small}
\end{center}
\end{table}


\section{Organização do Documento}
Para a apresentação desta proposta de trabalho de conclusão de curso, o presente documento está organizado como segue. Inicialmente, uma fundamendação teórica pode ser vista na Seção \ref{sec:fund_teorica}. Uma análise dos trabalhos relacionados encontra-se na Seção \ref{sec:trab_relac}. Na Seção \ref{sec:solucao_proposta} detalha-se uma solução proposta para a tarefa endereçada. Na Seção \ref{sec:resultados} estão os resultados obtidos. Finalmente, as considerações finais podem ser encontradas na Seção \ref{sec:consid_finais}.
