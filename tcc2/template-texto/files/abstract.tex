
  This work presents a proposal for estimating the age of viewers for content suggestion applications on \emph {Smart} TVs using machine learning techniques. Such a tool can be used in a variety of ways, including to facilitate the collection of information that contributes to a better content delivery experience, to the creation and control of custom settings, and to the implementation of more efficient parental control strategies. Nine training and test approaches are conducted using canonical CNNs architectures, such as LeNet, AlexNet, VGG-16 and SqueezeNet, whose output layers are submitted to an adaptation for a regression task. Although it is not possible to observe a progressive improvement, this work was conducted according to heuristic practices suggested by several authors from literature, considering models compatible with the state of the art. Results obtained showed that the best CNN for this task was LeNet with data augmentation, histogram equalization and activation function, which obtained a MAE of $9.98$. As future work, it is considered to address this problem as a classification task, as well as a more detailed analysis of the age groups with the highest and lowest error rates, in an attempt to obtain more appropriate estimators considering the characteristics of each group. Continuing to address this problem is of paramount importance from the practical point of view for the development of several content recommendation and parental control solutions on Smart TVs.

  \noindent \emph{\textbf{Keywords}: Age Estimation, Convolutional Neural Networks, Deep Learning}.
