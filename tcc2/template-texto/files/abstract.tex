
  This work presents a proposal for estimating the age of individuals using deep learning techniques. Such tool can be used in many ways. In smart TVs, for example, it can ease the collection of informations that can contribute to a better content delivery experience, to create and control custom settings, and to implement more efficient parental control strategies. In this work, the estimation of age of individuals using images of their faces is approached as a regression task. In order to address the referred task, nine approaches are conducted for training and testing different architectures of convolutional neural networks, which are submitted to parameters and hyperparameters variations. Although it is not possible to observe a progressive improvement, this work was conducted according to heuristic practices suggested by several authors from literature, considering models compatible with the state of the art. Obtained results showed that the best convolutional neural network for this task was LeNet which input images were submitted to data augmentation and histogram equalization techniques, using ReLU activation function. This model obtained a MAE of $9.98$. As future work, it is considered to address this problem as a classification task, as well as a more detailed analysis of the age groups with the highest and lowest error rates, in an attempt to obtain more appropriate estimators considering the characteristics of each group. Further addressing this problem is of paramount importance for the development of several solutions envolving the provision of automatically customized content.

  \noindent \emph{\textbf{Keywords}: Age Estimation, Convolutional Neural Networks, Deep Learning}.
