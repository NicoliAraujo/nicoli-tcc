%!TEX root = ../novoIndex.tex

  Este trabalho apresenta uma proposta para estimação de idade de telespectadores para aplicações de sugestão de conteúdo em \emph{Smart} TVs utilizando técnicas de \emph{deep learning}. Tal ferramenta pode ser utilizada de diversas maneiras, incluindo para facilitar a coleta de informações que contribuam para melhor experiência de provimento de conteúdo, criação e controle de configurações personalizadas e para a implementação de estratégias de controle parental mais eficientes. São conduzidas 9 abordagens de treino e teste utilizando arquiteturas canônicas de CNNs, a citar: LeNet, AlexNet, VGG-16 e SqueezeNet, submetidas a uma adaptação na camada de saída para uma tarefa de regressão. Apesar de não ser possível constatar uma progressiva melhora, o trabalho foi conduzido conforme práticas de natureza heurística sugeridas por diversos autores da literatura. Os resultados obtidos mostraram que a melhor CNN para esta tarefa foi LeNet com entrada de dados com \emph{data augmentation}, equalização por histograma e função de ativação \emph{ReLU}, obtendo-se um MAE de $9.98$ anos. Como trabalhos futuros considera-se a abordagem deste problema como uma tarefa de classificação, além de análise dos grupos de idade com maior e menor índice de erro, na tentativa de obter estimadores mais apropriados.  Continuar endereçando este problema é de suma importância do ponto de vista prático para o desenvolvimento de diversas soluções de recomendação de conteúdo e controle parental em \emph{Smart} TVs.


  \ \ \newline
  \noindent{\textbf{Palavras Chave}}: Estimação de Idade, Redes Neurais Artificiais Convolucionais, \emph{Deep Learning}.
