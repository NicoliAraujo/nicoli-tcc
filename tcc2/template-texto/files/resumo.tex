%!TEX root = ../novoIndex.tex

  Este trabalho apresenta uma proposta para estimação de idade de indivíduos utilizando técnicas de \emph{deep learning}. Tal ferramenta pode ser utilizada de diversas maneiras. Em \emph{Smart TVs}, por exemplo, pode facilitar a coleta de informações que contribuam para melhor experiência de provimento de conteúdo, criação e controle de configurações personalizadas e para a implementação de estratégias de controle parental mais eficientes. Neste trabalho, a estimação de idade de indivíduos a partir de imagens de face é tratada como tarefa de regressão. Para endereçá-la, são conduzidas nove abordagens de treino e teste de diferentes arquiteturas de redes neurais convolucionais, submetidas a variações de parâmetros e hiperparâmetros. Apesar de não ser possível constatar uma melhora progressiva, o trabalho foi conduzido conforme práticas de natureza heurística sugeridas por diversos autores da literatura. Os resultados obtidos mostraram que a melhor rede neural convolucional para esta tarefa foi LeNet cujas imagens de entrada foram submetidas a técnicas de \emph{data augmentation} e equalização por histograma, utilizando funções de ativação \emph{ReLU}. Este modelo obteve um MAE de $9.98$ anos. Como trabalhos futuros considera-se a abordagem deste problema como uma tarefa de classificação, além de análise dos grupos de idade com maior e menor índice de erro, na tentativa de obter estimadores mais apropriados. Continuar endereçando este problema é de suma importância do ponto de vista prático para o desenvolvimento de diversas soluções de personalização automática de conteúdo.


  \ \ \newline
  \noindent{\textbf{Palavras Chave}}: Estimação de Idade, Redes Neurais Convolucionais, \emph{Deep Learning}.
