%!TEX root = ../../sbc-template.tex

A realização de um projeto de mestrado desta natureza é justificada por várias razões. No contexto da interação entre telespectador e \emph{Smart} TV, um estimador de idade pode ser utilizado para facilitar a coleta de informações que contribuam para melhor experiência de provimento de conteúdo e de configurações personalizadas. Em particular, a estimação de idade dos telespectadores pode ser especialmente empregada na implementação de um controle parental mais eficiente, protegendo crianças e adolescentes de conteúdos inadequados à sua faixa etária.

Quanto ao provimento de propaganda direcionada, a estimação da idade pode contribuir para melhor adequação de vitrines e propagandas mostradas em telões distribuídos por shoppings a partir da idade dos indivíduos presentes naquele momento. Por exemplo, sabe-se que no almoço há uma maior circulação de jovens nos shoppings, que saem das escolas. Propagandas de filmes com público alvo mais jovem. Restaurantes que vendem bebidas alcóolicas e comida podem utilizar sistemas que vejam o público do momento para mostrar uma recomendação de produto a ser consumido.

A estimação da idade pode ser abordada de diversas maneiras. É possível realizar uma tarefa de regressão, em que a saída do modelo é um número real em um intervalo fechado condizente com o intervalo de idades de seres humanos. Pode-se também tratar como um problema de classificação, em que diversas quantidades de classes podem ser utilizadas. Por exemplo, jovem,  mediano ou adulto, ou intervalos de idade separados por dez anos. Dependendo da escolha de tarefa de aprendizado utilziada, a tarefa pode ficar mais complexa. É possível realizar uma comparação entre as técnicas de classificação e regressão para verificar qual é mais vantajosa.

\todo{revisar}
