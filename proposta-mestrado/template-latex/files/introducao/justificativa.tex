%!TEX root = ../../sbc-template.tex

A realização de um projeto de mestrado desta natureza é justificada por várias razões. No contexto da interação entre telespectador e \emph{Smart} TV, um estimador de idade pode ser utilizado para facilitar a coleta de informações que contribuam para melhor experiência de provimento de conteúdo e de configurações personalizadas. Em particular, a estimação de idade dos telespectadores pode ser especialmente empregada na implementação de um controle parental mais eficiente, protegendo crianças e adolescentes de conteúdos inadequados à sua faixa etária.

Quanto ao provimento de propaganda direcionada em vitrines e telões distribuídos em shoppings e centros comerciais, a estimação da idade de frequentadores pode ser utilizada juntamente com outras informações como dias, horários e corredores de pico ou histórico de compras de clientes, a fim de que lojas tenham maior conhecimento sobre seus consumidores, de maneira a oferecer ofertas e produtos mais adequados a cada público. As idades aferidas nos referidos locais de livre circulação também podem ser utilizadas para o direcionamento de propagandas governamentais que tenham público-alvo com idade específica, a exemplo de campanhas de vacinação, matrícula em escolas púbilicas, alistamento para serviço militar, entre outros.

No contexto do desenvolvimento de uma dissertação de mestrado, esta proposta é justificada pelo possibilidade de comparação de desempenho de diferentes abordagens para estimação de idade utilizando \emph{Deep Learning} encontradas na literatura. Este aspecto prevê a reprodutibilidade de resultados previamente descritos, e possibilita a aplicação de técnicas e modelos ainda não documentados para esta aplicação. Os resultados obtidos com tais experimentos podem vir a contribuir para o estado da arte neste ramo do conhecimento.
