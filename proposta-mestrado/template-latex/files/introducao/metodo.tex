%!TEX root = ../../sbc-template.tex
A metodologia para o desenvolvimento deste trabalho consiste na realização da \emph{contextualização bibliográfica sobre Deep Learning}, em especial contemplando os conceitos relativos às redes neurais convolucionais. Para tanto, considera-se a literatura desta área para que haja o entendimento das bases matemáticas deste modelo computacional, como funcionam, quais as características e as arquiteturras mais importantes. Neste estudo, além dos aspectos teóricos, serão considerados os ambientes de desenvolvimento, bibliotecas e outras tecnologias para implementação dos conceitos contemplados.

Os demais passos que compõem a metodologia deste trabalho baseiam-se no \emph{fluxo de atividades de machine learning} \cite{marsland2015machine}. Inicialmente, haverá a aquisição e o pré-processamento de imagens para \emph{consolidar uma base de dados} para esta tarefa de aprendizado. Nesta etapa, será considerada a literatura e uma base de dados já disponível e apropriadamente anotada, com licença livre de utilização.

A seguir, haverá a \emph{proposição de diferentes modelos de redes neurais convolucionais} para a tarefa de aprendizado considerada. Nesta etapa, serão elencados diferentes parâmetros e hiperparâmetros de configuração, bem como arquiteturas. Estes procedimentos visam consolidar um espaço de busca de modelos que possam endereçar a tarefa de maneira mais eficiente.

O próximo estágio consiste no \emph{treinamento das redes neurais convolucionais} para o problema em questão. Durante este processo, uma parte da base de dados será apresentada aos modelos para que haja o ajuste de pesos, compreendendo o aprendizado das características relevantes. O treinamento das redes ocorrerá utilizando computação em nuvem e computadores disponíveis no Laboratório de Sistemas Inteligentes (LSI), tendo em vista a infra-estrutura de hardware necessária para realizar este procedimento.

Segue-se então o \emph{teste das redes}, respeitando uma abordagem de validação cruzada e utilizando métricas de desempenho apropriadas. O objetivo desta fase consiste em aferir os modelos propostos e treinados quanto à sua capacidade de generalização.

Por fim, para identificação de um modelo mais adequado à esta tarefa, as \emph{métricas de desempenho serão comparadas} e os melhores modelos elencados a partir destes valores, apontando assim um estimador apropriado para o problema inicialmente considerado.

Alem destas atividades, há que se considerar a escrita e defesa do projeto final de dissertação de mestrado.
