%!TEX root = ../sbc-template.tex

A face humana detém diversos atributos que podem ser utilizados para caracterizar indivíduos. Os traços faciais provém informações como identidade, expressão facial, gênero, origem étnica e idade de uma pessoa. Quanto a esta última, seres humanos são capazes de determinar a idade de indivíduos entre 20 e 60 anos a partir da face com alta acurácia. Porém, há casos em que a checagem de idade realizada por humanos é inconveniente. Neste contexto, pode-se fazer uso de sistemas de determinação automática de idade, especialmente quando é necessário apenas saber a idade de um indivíduo, sem identificá-lo \cite{fu2010age}.

A estimação de idade consiste de rotular a imagem de uma face automaticamente com um número ou uma classe que corresponda a uma idade ou intervalo de idade do indivíduo portador da face fotografada. Estimadores de idade são particularmente úteis em aplicações em que é necessário ou desejável conhecer a idade do usuário para que seja entregue um produto ou serviço de maneira personalizada. Alguns exemplos incluem sistemas de recomendação de conteúdo, enfermeiras robóticas, propagandas direcionadas, entre outros. Há também situações em que a idade do indivíduo determina se haverá ou não o provimento de produto ou serviço, ou se sua entrada será permitida em algum ambiente. Exemplos destes casos incluem sistemas de checagem de idade na entrada de bares e boates, em máquinas de venda automática que contenham bebidas alcóolicas, e sistemas de controle parental \cite{fu2010age}.

No caso específico de \emph{Smart} TVs, por exemplo, é essencial que estes aparelhos sejam capazes de capturar o perfil e o interesse dos seus telespectadores a fim de oferecer uma experiência mais rica. A recomendação de conteúdo, por exemplo, pode levar em conta características individuais, tais como idade e gênero. Porém, se fornecidos de maneira habitual, via preenchimento de formulários, além de ser uma tarefa massante, podem não refletir de maneira realística o perfil individual dos vários usuários que podem estar à frente de uma \emph{Smart} TV em um determinado momento.

\emph{Smart} TVs possuem câmeras que podem ser habilitadas para aquisição de imagens daqueles que estão à frente do televisor, respeitadas as preferências de privacidade de cada usuário. É possível usá-las como entrada para sistemas inteligentes de identificação de características, cujas previsões podem ser aplicadas, por exemplo, na recomendação de conteúdo. No caso da idade, em particular, é possível usar estas informações para realizar um controle parental mais eficiente, protegendo crianças e adolescentes de conteúdos inadequados à sua faixa etária.\cite{Guardian:CameraSmartv}.

Diante do que foi exposto, esta proposta de projeto de mestrado considera o desenvolvimento de estratégias inteligentes, baseadas na utilização de técnicas de \emph{Deep Learning}, para estimação da idade de indivíduos a partir de fotografias faciais. Embora a estimação de outras características também pudesse ser realizada mediante a análise de fotografias faciais, desde gênero até a presença de doenças, optou-se pela idade por ser um atributo comum a todos os indivíduos, pelo potencial de aplicações, pela existência de bases de dados adequadamente rotuladas com este atributo e pelo menor potencial de infringência das searas privadas dos usuários.
